\documentclass[12pt,a4paper,notitlepage]{article}
\usepackage[utf8x]{inputenc}
\usepackage[a4paper,textwidth=17cm, top=2cm, bottom=3.5cm]{geometry}
\usepackage{eurosym}
%\usepackage{url}
\usepackage[T1]{fontenc}
\usepackage{ucs}
\usepackage{ngerman} 
\usepackage{setspace}
%\usepackage{fourier}
\usepackage{amssymb,amsmath}
\usepackage{wasysym}
\usepackage{amsthm}
%\usepackage{marvosym}
\usepackage{tabularx}
\usepackage{multicol}
\usepackage{hyperref}
\usepackage[pdftex]{graphicx,color}
\usepackage{todo}
\definecolor{p-green}{rgb}{0.12,0.57,0.11}
\newcommand{\bitem}{\item[--]}
\newcommand{\litem}[2]{\item[#1 --] #2}
\newcommand{\blitem}[3]{\item[#1 --] \texttt{#2} -- #3}
\newcommand{\gfo}{\grqq\ }
\newcommand{\gfu}{\glqq}
\newcommand{\zquote}[2]{\glqq #1\grqq\ (Z.\ #2)}
\newcommand{\pquote}[1]{\glqq #1\grqq}
\newcommand{\nquote}[2]{#1: \glqq #2\grqq}
\newcommand{\nwquote}[3]{#1 -- \emph{#2}: \glqq #3\grqq}
\newcommand{\nwyquote}[4]{#1 -- \emph{#2} (#3): \glqq #4\grqq}
\newcommand{\diff}{\mathrm{d}}
\renewcommand{\abstractname}{}
\definecolor{orange}{rgb}{1,0.6,0}
\definecolor{d-green}{rgb}{0,0.8,0}
\definecolor{pink}{rgb}{1,0,0.6}
\newcommand{\annot}[1]{\textcolor{red}{#1}}
\newcommand{\ecolor}[1]{\textcolor{pink}{#1}}
\newcommand{\aufgabe}[1]{\section*{\setcounter{section}{#1}Aufgabe #1}}
\newcommand{\re}{\text{Re}}
\newcommand{\im}{\text{Im}}
\onehalfspacing
\setlength{\parskip}{8pt plus4pt minus4pt}
\title{}
\begin{document}
\aufgabe{1}
\subsection*{a)}
\begin{align}
F_{Feder}(x)&=-ka\\
a&=\left(\sqrt{L^2+x^2}-\ell_0\right)\\
U(a)&=\frac{k}{2}a^2\\
\Leftrightarrow U(x)&=\frac{k}{2}\left(\sqrt{L^2+x^2}-\ell_0\right)^2\\
\end{align}
\subsubsection{Der Vollständigkeit halber: Erster Fehlschlag.}
\begin{align}
F_{Feder}&=-ka=-k\left(b-\ell_0\right)\\
b&=\sqrt{L^2+x^2}\\
F_{Feder}&=-k\left(\sqrt{L^2+x^2}-\ell_0\right)\\
\text{Kraft entlang }\vec x\text{: }F_x&=F_{Feder}\cdot\frac{x}{\sqrt{L^2+x^2}}\\
&=-kx\left(1-\frac{\ell_0}{\sqrt{L^2+x^2}}\right)\\
V(x)&=-\int F(x)\diff x\;\Big|\text{Annahme: }c=0\\
&=kx\left(\ell_0\text{asinh}\left(\frac{x}{\left|L\right|}\right)-x\right)
\end{align}
\subsection*{b)}
Siehe unten.
\begin{figure}[h!t]\begin{center}% GNUPLOT: LaTeX picture
\setlength{\unitlength}{0.240900pt}
\ifx\plotpoint\undefined\newsavebox{\plotpoint}\fi
\sbox{\plotpoint}{\rule[-0.200pt]{0.400pt}{0.400pt}}%
\begin{picture}(1500,900)(0,0)
\sbox{\plotpoint}{\rule[-0.200pt]{0.400pt}{0.400pt}}%
\put(151.0,131.0){\rule[-0.200pt]{4.818pt}{0.400pt}}
\put(131,131){\makebox(0,0)[r]{ 1}}
\put(1429.0,131.0){\rule[-0.200pt]{4.818pt}{0.400pt}}
\put(151.0,239.0){\rule[-0.200pt]{4.818pt}{0.400pt}}
\put(131,239){\makebox(0,0)[r]{ 2}}
\put(1429.0,239.0){\rule[-0.200pt]{4.818pt}{0.400pt}}
\put(151.0,346.0){\rule[-0.200pt]{4.818pt}{0.400pt}}
\put(131,346){\makebox(0,0)[r]{ 3}}
\put(1429.0,346.0){\rule[-0.200pt]{4.818pt}{0.400pt}}
\put(151.0,454.0){\rule[-0.200pt]{4.818pt}{0.400pt}}
\put(131,454){\makebox(0,0)[r]{ 4}}
\put(1429.0,454.0){\rule[-0.200pt]{4.818pt}{0.400pt}}
\put(151.0,561.0){\rule[-0.200pt]{4.818pt}{0.400pt}}
\put(131,561){\makebox(0,0)[r]{ 5}}
\put(1429.0,561.0){\rule[-0.200pt]{4.818pt}{0.400pt}}
\put(151.0,669.0){\rule[-0.200pt]{4.818pt}{0.400pt}}
\put(131,669){\makebox(0,0)[r]{ 6}}
\put(1429.0,669.0){\rule[-0.200pt]{4.818pt}{0.400pt}}
\put(151.0,776.0){\rule[-0.200pt]{4.818pt}{0.400pt}}
\put(131,776){\makebox(0,0)[r]{ 7}}
\put(1429.0,776.0){\rule[-0.200pt]{4.818pt}{0.400pt}}
\put(151.0,131.0){\rule[-0.200pt]{0.400pt}{4.818pt}}
\put(151,90){\makebox(0,0){-3}}
\put(151.0,756.0){\rule[-0.200pt]{0.400pt}{4.818pt}}
\put(367.0,131.0){\rule[-0.200pt]{0.400pt}{4.818pt}}
\put(367,90){\makebox(0,0){-2}}
\put(367.0,756.0){\rule[-0.200pt]{0.400pt}{4.818pt}}
\put(584.0,131.0){\rule[-0.200pt]{0.400pt}{4.818pt}}
\put(584,90){\makebox(0,0){-1}}
\put(584.0,756.0){\rule[-0.200pt]{0.400pt}{4.818pt}}
\put(800.0,131.0){\rule[-0.200pt]{0.400pt}{4.818pt}}
\put(800,90){\makebox(0,0){ 0}}
\put(800.0,756.0){\rule[-0.200pt]{0.400pt}{4.818pt}}
\put(1016.0,131.0){\rule[-0.200pt]{0.400pt}{4.818pt}}
\put(1016,90){\makebox(0,0){ 1}}
\put(1016.0,756.0){\rule[-0.200pt]{0.400pt}{4.818pt}}
\put(1233.0,131.0){\rule[-0.200pt]{0.400pt}{4.818pt}}
\put(1233,90){\makebox(0,0){ 2}}
\put(1233.0,756.0){\rule[-0.200pt]{0.400pt}{4.818pt}}
\put(1449.0,131.0){\rule[-0.200pt]{0.400pt}{4.818pt}}
\put(1449,90){\makebox(0,0){ 3}}
\put(1449.0,756.0){\rule[-0.200pt]{0.400pt}{4.818pt}}
\put(151.0,131.0){\rule[-0.200pt]{0.400pt}{155.380pt}}
\put(151.0,131.0){\rule[-0.200pt]{312.688pt}{0.400pt}}
\put(1449.0,131.0){\rule[-0.200pt]{0.400pt}{155.380pt}}
\put(151.0,776.0){\rule[-0.200pt]{312.688pt}{0.400pt}}
\put(50,453){\makebox(0,0){V(x)}}
\put(800,29){\makebox(0,0){x}}
\put(800,838){\makebox(0,0){$l_0<L$}}
\put(1289,736){\makebox(0,0)[r]{f(x)}}
\put(1309.0,736.0){\rule[-0.200pt]{24.090pt}{0.400pt}}
\put(151,753){\usebox{\plotpoint}}
\multiput(151.58,749.01)(0.493,-1.091){23}{\rule{0.119pt}{0.962pt}}
\multiput(150.17,751.00)(13.000,-26.004){2}{\rule{0.400pt}{0.481pt}}
\multiput(164.58,721.14)(0.493,-1.052){23}{\rule{0.119pt}{0.931pt}}
\multiput(163.17,723.07)(13.000,-25.068){2}{\rule{0.400pt}{0.465pt}}
\multiput(177.58,694.26)(0.493,-1.012){23}{\rule{0.119pt}{0.900pt}}
\multiput(176.17,696.13)(13.000,-24.132){2}{\rule{0.400pt}{0.450pt}}
\multiput(190.58,668.26)(0.493,-1.012){23}{\rule{0.119pt}{0.900pt}}
\multiput(189.17,670.13)(13.000,-24.132){2}{\rule{0.400pt}{0.450pt}}
\multiput(203.58,642.62)(0.494,-0.901){25}{\rule{0.119pt}{0.814pt}}
\multiput(202.17,644.31)(14.000,-23.310){2}{\rule{0.400pt}{0.407pt}}
\multiput(217.58,617.52)(0.493,-0.933){23}{\rule{0.119pt}{0.838pt}}
\multiput(216.17,619.26)(13.000,-22.260){2}{\rule{0.400pt}{0.419pt}}
\multiput(230.58,593.52)(0.493,-0.933){23}{\rule{0.119pt}{0.838pt}}
\multiput(229.17,595.26)(13.000,-22.260){2}{\rule{0.400pt}{0.419pt}}
\multiput(243.58,569.65)(0.493,-0.893){23}{\rule{0.119pt}{0.808pt}}
\multiput(242.17,571.32)(13.000,-21.324){2}{\rule{0.400pt}{0.404pt}}
\multiput(256.58,546.77)(0.493,-0.853){23}{\rule{0.119pt}{0.777pt}}
\multiput(255.17,548.39)(13.000,-20.387){2}{\rule{0.400pt}{0.388pt}}
\multiput(269.58,524.77)(0.493,-0.853){23}{\rule{0.119pt}{0.777pt}}
\multiput(268.17,526.39)(13.000,-20.387){2}{\rule{0.400pt}{0.388pt}}
\multiput(282.58,503.03)(0.493,-0.774){23}{\rule{0.119pt}{0.715pt}}
\multiput(281.17,504.52)(13.000,-18.515){2}{\rule{0.400pt}{0.358pt}}
\multiput(295.58,482.90)(0.493,-0.814){23}{\rule{0.119pt}{0.746pt}}
\multiput(294.17,484.45)(13.000,-19.451){2}{\rule{0.400pt}{0.373pt}}
\multiput(308.58,462.16)(0.493,-0.734){23}{\rule{0.119pt}{0.685pt}}
\multiput(307.17,463.58)(13.000,-17.579){2}{\rule{0.400pt}{0.342pt}}
\multiput(321.58,443.33)(0.494,-0.680){25}{\rule{0.119pt}{0.643pt}}
\multiput(320.17,444.67)(14.000,-17.666){2}{\rule{0.400pt}{0.321pt}}
\multiput(335.58,424.29)(0.493,-0.695){23}{\rule{0.119pt}{0.654pt}}
\multiput(334.17,425.64)(13.000,-16.643){2}{\rule{0.400pt}{0.327pt}}
\multiput(348.58,406.29)(0.493,-0.695){23}{\rule{0.119pt}{0.654pt}}
\multiput(347.17,407.64)(13.000,-16.643){2}{\rule{0.400pt}{0.327pt}}
\multiput(361.58,388.54)(0.493,-0.616){23}{\rule{0.119pt}{0.592pt}}
\multiput(360.17,389.77)(13.000,-14.771){2}{\rule{0.400pt}{0.296pt}}
\multiput(374.58,372.41)(0.493,-0.655){23}{\rule{0.119pt}{0.623pt}}
\multiput(373.17,373.71)(13.000,-15.707){2}{\rule{0.400pt}{0.312pt}}
\multiput(387.58,355.67)(0.493,-0.576){23}{\rule{0.119pt}{0.562pt}}
\multiput(386.17,356.83)(13.000,-13.834){2}{\rule{0.400pt}{0.281pt}}
\multiput(400.58,340.67)(0.493,-0.576){23}{\rule{0.119pt}{0.562pt}}
\multiput(399.17,341.83)(13.000,-13.834){2}{\rule{0.400pt}{0.281pt}}
\multiput(413.58,325.80)(0.493,-0.536){23}{\rule{0.119pt}{0.531pt}}
\multiput(412.17,326.90)(13.000,-12.898){2}{\rule{0.400pt}{0.265pt}}
\multiput(426.58,311.80)(0.493,-0.536){23}{\rule{0.119pt}{0.531pt}}
\multiput(425.17,312.90)(13.000,-12.898){2}{\rule{0.400pt}{0.265pt}}
\multiput(439.00,298.92)(0.536,-0.493){23}{\rule{0.531pt}{0.119pt}}
\multiput(439.00,299.17)(12.898,-13.000){2}{\rule{0.265pt}{0.400pt}}
\multiput(453.00,285.92)(0.497,-0.493){23}{\rule{0.500pt}{0.119pt}}
\multiput(453.00,286.17)(11.962,-13.000){2}{\rule{0.250pt}{0.400pt}}
\multiput(466.00,272.92)(0.539,-0.492){21}{\rule{0.533pt}{0.119pt}}
\multiput(466.00,273.17)(11.893,-12.000){2}{\rule{0.267pt}{0.400pt}}
\multiput(479.00,260.92)(0.590,-0.492){19}{\rule{0.573pt}{0.118pt}}
\multiput(479.00,261.17)(11.811,-11.000){2}{\rule{0.286pt}{0.400pt}}
\multiput(492.00,249.92)(0.590,-0.492){19}{\rule{0.573pt}{0.118pt}}
\multiput(492.00,250.17)(11.811,-11.000){2}{\rule{0.286pt}{0.400pt}}
\multiput(505.00,238.92)(0.652,-0.491){17}{\rule{0.620pt}{0.118pt}}
\multiput(505.00,239.17)(11.713,-10.000){2}{\rule{0.310pt}{0.400pt}}
\multiput(518.00,228.93)(0.728,-0.489){15}{\rule{0.678pt}{0.118pt}}
\multiput(518.00,229.17)(11.593,-9.000){2}{\rule{0.339pt}{0.400pt}}
\multiput(531.00,219.93)(0.728,-0.489){15}{\rule{0.678pt}{0.118pt}}
\multiput(531.00,220.17)(11.593,-9.000){2}{\rule{0.339pt}{0.400pt}}
\multiput(544.00,210.93)(0.728,-0.489){15}{\rule{0.678pt}{0.118pt}}
\multiput(544.00,211.17)(11.593,-9.000){2}{\rule{0.339pt}{0.400pt}}
\multiput(557.00,201.93)(0.890,-0.488){13}{\rule{0.800pt}{0.117pt}}
\multiput(557.00,202.17)(12.340,-8.000){2}{\rule{0.400pt}{0.400pt}}
\multiput(571.00,193.93)(0.950,-0.485){11}{\rule{0.843pt}{0.117pt}}
\multiput(571.00,194.17)(11.251,-7.000){2}{\rule{0.421pt}{0.400pt}}
\multiput(584.00,186.93)(0.950,-0.485){11}{\rule{0.843pt}{0.117pt}}
\multiput(584.00,187.17)(11.251,-7.000){2}{\rule{0.421pt}{0.400pt}}
\multiput(597.00,179.93)(0.950,-0.485){11}{\rule{0.843pt}{0.117pt}}
\multiput(597.00,180.17)(11.251,-7.000){2}{\rule{0.421pt}{0.400pt}}
\multiput(610.00,172.93)(1.123,-0.482){9}{\rule{0.967pt}{0.116pt}}
\multiput(610.00,173.17)(10.994,-6.000){2}{\rule{0.483pt}{0.400pt}}
\multiput(623.00,166.93)(1.378,-0.477){7}{\rule{1.140pt}{0.115pt}}
\multiput(623.00,167.17)(10.634,-5.000){2}{\rule{0.570pt}{0.400pt}}
\multiput(636.00,161.93)(1.378,-0.477){7}{\rule{1.140pt}{0.115pt}}
\multiput(636.00,162.17)(10.634,-5.000){2}{\rule{0.570pt}{0.400pt}}
\multiput(649.00,156.93)(1.378,-0.477){7}{\rule{1.140pt}{0.115pt}}
\multiput(649.00,157.17)(10.634,-5.000){2}{\rule{0.570pt}{0.400pt}}
\multiput(662.00,151.94)(1.797,-0.468){5}{\rule{1.400pt}{0.113pt}}
\multiput(662.00,152.17)(10.094,-4.000){2}{\rule{0.700pt}{0.400pt}}
\multiput(675.00,147.94)(1.943,-0.468){5}{\rule{1.500pt}{0.113pt}}
\multiput(675.00,148.17)(10.887,-4.000){2}{\rule{0.750pt}{0.400pt}}
\multiput(689.00,143.95)(2.695,-0.447){3}{\rule{1.833pt}{0.108pt}}
\multiput(689.00,144.17)(9.195,-3.000){2}{\rule{0.917pt}{0.400pt}}
\multiput(702.00,140.95)(2.695,-0.447){3}{\rule{1.833pt}{0.108pt}}
\multiput(702.00,141.17)(9.195,-3.000){2}{\rule{0.917pt}{0.400pt}}
\put(715,137.17){\rule{2.700pt}{0.400pt}}
\multiput(715.00,138.17)(7.396,-2.000){2}{\rule{1.350pt}{0.400pt}}
\put(728,135.17){\rule{2.700pt}{0.400pt}}
\multiput(728.00,136.17)(7.396,-2.000){2}{\rule{1.350pt}{0.400pt}}
\put(741,133.17){\rule{2.700pt}{0.400pt}}
\multiput(741.00,134.17)(7.396,-2.000){2}{\rule{1.350pt}{0.400pt}}
\put(754,131.67){\rule{3.132pt}{0.400pt}}
\multiput(754.00,132.17)(6.500,-1.000){2}{\rule{1.566pt}{0.400pt}}
\put(767,130.67){\rule{3.132pt}{0.400pt}}
\multiput(767.00,131.17)(6.500,-1.000){2}{\rule{1.566pt}{0.400pt}}
\put(820,130.67){\rule{3.132pt}{0.400pt}}
\multiput(820.00,130.17)(6.500,1.000){2}{\rule{1.566pt}{0.400pt}}
\put(833,131.67){\rule{3.132pt}{0.400pt}}
\multiput(833.00,131.17)(6.500,1.000){2}{\rule{1.566pt}{0.400pt}}
\put(846,133.17){\rule{2.700pt}{0.400pt}}
\multiput(846.00,132.17)(7.396,2.000){2}{\rule{1.350pt}{0.400pt}}
\put(859,135.17){\rule{2.700pt}{0.400pt}}
\multiput(859.00,134.17)(7.396,2.000){2}{\rule{1.350pt}{0.400pt}}
\put(872,137.17){\rule{2.700pt}{0.400pt}}
\multiput(872.00,136.17)(7.396,2.000){2}{\rule{1.350pt}{0.400pt}}
\multiput(885.00,139.61)(2.695,0.447){3}{\rule{1.833pt}{0.108pt}}
\multiput(885.00,138.17)(9.195,3.000){2}{\rule{0.917pt}{0.400pt}}
\multiput(898.00,142.61)(2.695,0.447){3}{\rule{1.833pt}{0.108pt}}
\multiput(898.00,141.17)(9.195,3.000){2}{\rule{0.917pt}{0.400pt}}
\multiput(911.00,145.60)(1.943,0.468){5}{\rule{1.500pt}{0.113pt}}
\multiput(911.00,144.17)(10.887,4.000){2}{\rule{0.750pt}{0.400pt}}
\multiput(925.00,149.60)(1.797,0.468){5}{\rule{1.400pt}{0.113pt}}
\multiput(925.00,148.17)(10.094,4.000){2}{\rule{0.700pt}{0.400pt}}
\multiput(938.00,153.59)(1.378,0.477){7}{\rule{1.140pt}{0.115pt}}
\multiput(938.00,152.17)(10.634,5.000){2}{\rule{0.570pt}{0.400pt}}
\multiput(951.00,158.59)(1.378,0.477){7}{\rule{1.140pt}{0.115pt}}
\multiput(951.00,157.17)(10.634,5.000){2}{\rule{0.570pt}{0.400pt}}
\multiput(964.00,163.59)(1.378,0.477){7}{\rule{1.140pt}{0.115pt}}
\multiput(964.00,162.17)(10.634,5.000){2}{\rule{0.570pt}{0.400pt}}
\multiput(977.00,168.59)(1.123,0.482){9}{\rule{0.967pt}{0.116pt}}
\multiput(977.00,167.17)(10.994,6.000){2}{\rule{0.483pt}{0.400pt}}
\multiput(990.00,174.59)(0.950,0.485){11}{\rule{0.843pt}{0.117pt}}
\multiput(990.00,173.17)(11.251,7.000){2}{\rule{0.421pt}{0.400pt}}
\multiput(1003.00,181.59)(0.950,0.485){11}{\rule{0.843pt}{0.117pt}}
\multiput(1003.00,180.17)(11.251,7.000){2}{\rule{0.421pt}{0.400pt}}
\multiput(1016.00,188.59)(0.950,0.485){11}{\rule{0.843pt}{0.117pt}}
\multiput(1016.00,187.17)(11.251,7.000){2}{\rule{0.421pt}{0.400pt}}
\multiput(1029.00,195.59)(0.890,0.488){13}{\rule{0.800pt}{0.117pt}}
\multiput(1029.00,194.17)(12.340,8.000){2}{\rule{0.400pt}{0.400pt}}
\multiput(1043.00,203.59)(0.728,0.489){15}{\rule{0.678pt}{0.118pt}}
\multiput(1043.00,202.17)(11.593,9.000){2}{\rule{0.339pt}{0.400pt}}
\multiput(1056.00,212.59)(0.728,0.489){15}{\rule{0.678pt}{0.118pt}}
\multiput(1056.00,211.17)(11.593,9.000){2}{\rule{0.339pt}{0.400pt}}
\multiput(1069.00,221.59)(0.728,0.489){15}{\rule{0.678pt}{0.118pt}}
\multiput(1069.00,220.17)(11.593,9.000){2}{\rule{0.339pt}{0.400pt}}
\multiput(1082.00,230.58)(0.652,0.491){17}{\rule{0.620pt}{0.118pt}}
\multiput(1082.00,229.17)(11.713,10.000){2}{\rule{0.310pt}{0.400pt}}
\multiput(1095.00,240.58)(0.590,0.492){19}{\rule{0.573pt}{0.118pt}}
\multiput(1095.00,239.17)(11.811,11.000){2}{\rule{0.286pt}{0.400pt}}
\multiput(1108.00,251.58)(0.590,0.492){19}{\rule{0.573pt}{0.118pt}}
\multiput(1108.00,250.17)(11.811,11.000){2}{\rule{0.286pt}{0.400pt}}
\multiput(1121.00,262.58)(0.539,0.492){21}{\rule{0.533pt}{0.119pt}}
\multiput(1121.00,261.17)(11.893,12.000){2}{\rule{0.267pt}{0.400pt}}
\multiput(1134.00,274.58)(0.497,0.493){23}{\rule{0.500pt}{0.119pt}}
\multiput(1134.00,273.17)(11.962,13.000){2}{\rule{0.250pt}{0.400pt}}
\multiput(1147.00,287.58)(0.536,0.493){23}{\rule{0.531pt}{0.119pt}}
\multiput(1147.00,286.17)(12.898,13.000){2}{\rule{0.265pt}{0.400pt}}
\multiput(1161.58,300.00)(0.493,0.536){23}{\rule{0.119pt}{0.531pt}}
\multiput(1160.17,300.00)(13.000,12.898){2}{\rule{0.400pt}{0.265pt}}
\multiput(1174.58,314.00)(0.493,0.536){23}{\rule{0.119pt}{0.531pt}}
\multiput(1173.17,314.00)(13.000,12.898){2}{\rule{0.400pt}{0.265pt}}
\multiput(1187.58,328.00)(0.493,0.576){23}{\rule{0.119pt}{0.562pt}}
\multiput(1186.17,328.00)(13.000,13.834){2}{\rule{0.400pt}{0.281pt}}
\multiput(1200.58,343.00)(0.493,0.576){23}{\rule{0.119pt}{0.562pt}}
\multiput(1199.17,343.00)(13.000,13.834){2}{\rule{0.400pt}{0.281pt}}
\multiput(1213.58,358.00)(0.493,0.655){23}{\rule{0.119pt}{0.623pt}}
\multiput(1212.17,358.00)(13.000,15.707){2}{\rule{0.400pt}{0.312pt}}
\multiput(1226.58,375.00)(0.493,0.616){23}{\rule{0.119pt}{0.592pt}}
\multiput(1225.17,375.00)(13.000,14.771){2}{\rule{0.400pt}{0.296pt}}
\multiput(1239.58,391.00)(0.493,0.695){23}{\rule{0.119pt}{0.654pt}}
\multiput(1238.17,391.00)(13.000,16.643){2}{\rule{0.400pt}{0.327pt}}
\multiput(1252.58,409.00)(0.493,0.695){23}{\rule{0.119pt}{0.654pt}}
\multiput(1251.17,409.00)(13.000,16.643){2}{\rule{0.400pt}{0.327pt}}
\multiput(1265.58,427.00)(0.494,0.680){25}{\rule{0.119pt}{0.643pt}}
\multiput(1264.17,427.00)(14.000,17.666){2}{\rule{0.400pt}{0.321pt}}
\multiput(1279.58,446.00)(0.493,0.734){23}{\rule{0.119pt}{0.685pt}}
\multiput(1278.17,446.00)(13.000,17.579){2}{\rule{0.400pt}{0.342pt}}
\multiput(1292.58,465.00)(0.493,0.814){23}{\rule{0.119pt}{0.746pt}}
\multiput(1291.17,465.00)(13.000,19.451){2}{\rule{0.400pt}{0.373pt}}
\multiput(1305.58,486.00)(0.493,0.774){23}{\rule{0.119pt}{0.715pt}}
\multiput(1304.17,486.00)(13.000,18.515){2}{\rule{0.400pt}{0.358pt}}
\multiput(1318.58,506.00)(0.493,0.853){23}{\rule{0.119pt}{0.777pt}}
\multiput(1317.17,506.00)(13.000,20.387){2}{\rule{0.400pt}{0.388pt}}
\multiput(1331.58,528.00)(0.493,0.853){23}{\rule{0.119pt}{0.777pt}}
\multiput(1330.17,528.00)(13.000,20.387){2}{\rule{0.400pt}{0.388pt}}
\multiput(1344.58,550.00)(0.493,0.893){23}{\rule{0.119pt}{0.808pt}}
\multiput(1343.17,550.00)(13.000,21.324){2}{\rule{0.400pt}{0.404pt}}
\multiput(1357.58,573.00)(0.493,0.933){23}{\rule{0.119pt}{0.838pt}}
\multiput(1356.17,573.00)(13.000,22.260){2}{\rule{0.400pt}{0.419pt}}
\multiput(1370.58,597.00)(0.493,0.933){23}{\rule{0.119pt}{0.838pt}}
\multiput(1369.17,597.00)(13.000,22.260){2}{\rule{0.400pt}{0.419pt}}
\multiput(1383.58,621.00)(0.494,0.901){25}{\rule{0.119pt}{0.814pt}}
\multiput(1382.17,621.00)(14.000,23.310){2}{\rule{0.400pt}{0.407pt}}
\multiput(1397.58,646.00)(0.493,1.012){23}{\rule{0.119pt}{0.900pt}}
\multiput(1396.17,646.00)(13.000,24.132){2}{\rule{0.400pt}{0.450pt}}
\multiput(1410.58,672.00)(0.493,1.012){23}{\rule{0.119pt}{0.900pt}}
\multiput(1409.17,672.00)(13.000,24.132){2}{\rule{0.400pt}{0.450pt}}
\multiput(1423.58,698.00)(0.493,1.052){23}{\rule{0.119pt}{0.931pt}}
\multiput(1422.17,698.00)(13.000,25.068){2}{\rule{0.400pt}{0.465pt}}
\multiput(1436.58,725.00)(0.493,1.091){23}{\rule{0.119pt}{0.962pt}}
\multiput(1435.17,725.00)(13.000,26.004){2}{\rule{0.400pt}{0.481pt}}
\put(780.0,131.0){\rule[-0.200pt]{9.636pt}{0.400pt}}
\put(151.0,131.0){\rule[-0.200pt]{0.400pt}{155.380pt}}
\put(151.0,131.0){\rule[-0.200pt]{312.688pt}{0.400pt}}
\put(1449.0,131.0){\rule[-0.200pt]{0.400pt}{155.380pt}}
\put(151.0,776.0){\rule[-0.200pt]{312.688pt}{0.400pt}}
\end{picture}
\end{center}\end{figure}
\begin{figure}[h!t]\begin{center}% GNUPLOT: LaTeX picture
\setlength{\unitlength}{0.240900pt}
\ifx\plotpoint\undefined\newsavebox{\plotpoint}\fi
\sbox{\plotpoint}{\rule[-0.200pt]{0.400pt}{0.400pt}}%
\begin{picture}(1500,900)(0,0)
\sbox{\plotpoint}{\rule[-0.200pt]{0.400pt}{0.400pt}}%
\put(191.0,131.0){\rule[-0.200pt]{4.818pt}{0.400pt}}
\put(171,131){\makebox(0,0)[r]{ 0}}
\put(1429.0,131.0){\rule[-0.200pt]{4.818pt}{0.400pt}}
\put(191.0,223.0){\rule[-0.200pt]{4.818pt}{0.400pt}}
\put(171,223){\makebox(0,0)[r]{ 0.2}}
\put(1429.0,223.0){\rule[-0.200pt]{4.818pt}{0.400pt}}
\put(191.0,315.0){\rule[-0.200pt]{4.818pt}{0.400pt}}
\put(171,315){\makebox(0,0)[r]{ 0.4}}
\put(1429.0,315.0){\rule[-0.200pt]{4.818pt}{0.400pt}}
\put(191.0,407.0){\rule[-0.200pt]{4.818pt}{0.400pt}}
\put(171,407){\makebox(0,0)[r]{ 0.6}}
\put(1429.0,407.0){\rule[-0.200pt]{4.818pt}{0.400pt}}
\put(191.0,500.0){\rule[-0.200pt]{4.818pt}{0.400pt}}
\put(171,500){\makebox(0,0)[r]{ 0.8}}
\put(1429.0,500.0){\rule[-0.200pt]{4.818pt}{0.400pt}}
\put(191.0,592.0){\rule[-0.200pt]{4.818pt}{0.400pt}}
\put(171,592){\makebox(0,0)[r]{ 1}}
\put(1429.0,592.0){\rule[-0.200pt]{4.818pt}{0.400pt}}
\put(191.0,684.0){\rule[-0.200pt]{4.818pt}{0.400pt}}
\put(171,684){\makebox(0,0)[r]{ 1.2}}
\put(1429.0,684.0){\rule[-0.200pt]{4.818pt}{0.400pt}}
\put(191.0,776.0){\rule[-0.200pt]{4.818pt}{0.400pt}}
\put(171,776){\makebox(0,0)[r]{ 1.4}}
\put(1429.0,776.0){\rule[-0.200pt]{4.818pt}{0.400pt}}
\put(191.0,131.0){\rule[-0.200pt]{0.400pt}{4.818pt}}
\put(191,90){\makebox(0,0){-3}}
\put(191.0,756.0){\rule[-0.200pt]{0.400pt}{4.818pt}}
\put(401.0,131.0){\rule[-0.200pt]{0.400pt}{4.818pt}}
\put(401,90){\makebox(0,0){-2}}
\put(401.0,756.0){\rule[-0.200pt]{0.400pt}{4.818pt}}
\put(610.0,131.0){\rule[-0.200pt]{0.400pt}{4.818pt}}
\put(610,90){\makebox(0,0){-1}}
\put(610.0,756.0){\rule[-0.200pt]{0.400pt}{4.818pt}}
\put(820.0,131.0){\rule[-0.200pt]{0.400pt}{4.818pt}}
\put(820,90){\makebox(0,0){ 0}}
\put(820.0,756.0){\rule[-0.200pt]{0.400pt}{4.818pt}}
\put(1030.0,131.0){\rule[-0.200pt]{0.400pt}{4.818pt}}
\put(1030,90){\makebox(0,0){ 1}}
\put(1030.0,756.0){\rule[-0.200pt]{0.400pt}{4.818pt}}
\put(1239.0,131.0){\rule[-0.200pt]{0.400pt}{4.818pt}}
\put(1239,90){\makebox(0,0){ 2}}
\put(1239.0,756.0){\rule[-0.200pt]{0.400pt}{4.818pt}}
\put(1449.0,131.0){\rule[-0.200pt]{0.400pt}{4.818pt}}
\put(1449,90){\makebox(0,0){ 3}}
\put(1449.0,756.0){\rule[-0.200pt]{0.400pt}{4.818pt}}
\put(191.0,131.0){\rule[-0.200pt]{0.400pt}{155.380pt}}
\put(191.0,131.0){\rule[-0.200pt]{303.052pt}{0.400pt}}
\put(1449.0,131.0){\rule[-0.200pt]{0.400pt}{155.380pt}}
\put(191.0,776.0){\rule[-0.200pt]{303.052pt}{0.400pt}}
\put(50,453){\makebox(0,0){V(x)}}
\put(820,29){\makebox(0,0){x}}
\put(820,838){\makebox(0,0){$l_0>L$}}
\put(1289,736){\makebox(0,0)[r]{f(x)}}
\put(1309.0,736.0){\rule[-0.200pt]{24.090pt}{0.400pt}}
\put(191,753){\usebox{\plotpoint}}
\multiput(191.58,744.92)(0.493,-2.360){23}{\rule{0.119pt}{1.946pt}}
\multiput(190.17,748.96)(13.000,-55.961){2}{\rule{0.400pt}{0.973pt}}
\multiput(204.58,684.84)(0.492,-2.392){21}{\rule{0.119pt}{1.967pt}}
\multiput(203.17,688.92)(12.000,-51.918){2}{\rule{0.400pt}{0.983pt}}
\multiput(216.58,629.69)(0.493,-2.122){23}{\rule{0.119pt}{1.762pt}}
\multiput(215.17,633.34)(13.000,-50.344){2}{\rule{0.400pt}{0.881pt}}
\multiput(229.58,576.07)(0.493,-2.003){23}{\rule{0.119pt}{1.669pt}}
\multiput(228.17,579.54)(13.000,-47.535){2}{\rule{0.400pt}{0.835pt}}
\multiput(242.58,525.58)(0.493,-1.845){23}{\rule{0.119pt}{1.546pt}}
\multiput(241.17,528.79)(13.000,-43.791){2}{\rule{0.400pt}{0.773pt}}
\multiput(255.58,478.50)(0.492,-1.875){21}{\rule{0.119pt}{1.567pt}}
\multiput(254.17,481.75)(12.000,-40.748){2}{\rule{0.400pt}{0.783pt}}
\multiput(267.58,435.22)(0.493,-1.646){23}{\rule{0.119pt}{1.392pt}}
\multiput(266.17,438.11)(13.000,-39.110){2}{\rule{0.400pt}{0.696pt}}
\multiput(280.58,393.73)(0.493,-1.488){23}{\rule{0.119pt}{1.269pt}}
\multiput(279.17,396.37)(13.000,-35.366){2}{\rule{0.400pt}{0.635pt}}
\multiput(293.58,355.74)(0.492,-1.487){21}{\rule{0.119pt}{1.267pt}}
\multiput(292.17,358.37)(12.000,-32.371){2}{\rule{0.400pt}{0.633pt}}
\multiput(305.58,321.50)(0.493,-1.250){23}{\rule{0.119pt}{1.085pt}}
\multiput(304.17,323.75)(13.000,-29.749){2}{\rule{0.400pt}{0.542pt}}
\multiput(318.58,289.88)(0.493,-1.131){23}{\rule{0.119pt}{0.992pt}}
\multiput(317.17,291.94)(13.000,-26.940){2}{\rule{0.400pt}{0.496pt}}
\multiput(331.58,260.85)(0.492,-1.142){21}{\rule{0.119pt}{1.000pt}}
\multiput(330.17,262.92)(12.000,-24.924){2}{\rule{0.400pt}{0.500pt}}
\multiput(343.58,234.65)(0.493,-0.893){23}{\rule{0.119pt}{0.808pt}}
\multiput(342.17,236.32)(13.000,-21.324){2}{\rule{0.400pt}{0.404pt}}
\multiput(356.58,212.03)(0.493,-0.774){23}{\rule{0.119pt}{0.715pt}}
\multiput(355.17,213.52)(13.000,-18.515){2}{\rule{0.400pt}{0.358pt}}
\multiput(369.58,192.41)(0.493,-0.655){23}{\rule{0.119pt}{0.623pt}}
\multiput(368.17,193.71)(13.000,-15.707){2}{\rule{0.400pt}{0.312pt}}
\multiput(382.58,175.51)(0.492,-0.625){21}{\rule{0.119pt}{0.600pt}}
\multiput(381.17,176.75)(12.000,-13.755){2}{\rule{0.400pt}{0.300pt}}
\multiput(394.00,161.92)(0.539,-0.492){21}{\rule{0.533pt}{0.119pt}}
\multiput(394.00,162.17)(11.893,-12.000){2}{\rule{0.267pt}{0.400pt}}
\multiput(407.00,149.93)(0.728,-0.489){15}{\rule{0.678pt}{0.118pt}}
\multiput(407.00,150.17)(11.593,-9.000){2}{\rule{0.339pt}{0.400pt}}
\multiput(420.00,140.93)(1.033,-0.482){9}{\rule{0.900pt}{0.116pt}}
\multiput(420.00,141.17)(10.132,-6.000){2}{\rule{0.450pt}{0.400pt}}
\multiput(432.00,134.94)(1.797,-0.468){5}{\rule{1.400pt}{0.113pt}}
\multiput(432.00,135.17)(10.094,-4.000){2}{\rule{0.700pt}{0.400pt}}
\put(445,130.67){\rule{3.132pt}{0.400pt}}
\multiput(445.00,131.17)(6.500,-1.000){2}{\rule{1.566pt}{0.400pt}}
\put(458,130.67){\rule{3.132pt}{0.400pt}}
\multiput(458.00,130.17)(6.500,1.000){2}{\rule{1.566pt}{0.400pt}}
\multiput(471.00,132.60)(1.651,0.468){5}{\rule{1.300pt}{0.113pt}}
\multiput(471.00,131.17)(9.302,4.000){2}{\rule{0.650pt}{0.400pt}}
\multiput(483.00,136.59)(0.950,0.485){11}{\rule{0.843pt}{0.117pt}}
\multiput(483.00,135.17)(11.251,7.000){2}{\rule{0.421pt}{0.400pt}}
\multiput(496.00,143.59)(0.824,0.488){13}{\rule{0.750pt}{0.117pt}}
\multiput(496.00,142.17)(11.443,8.000){2}{\rule{0.375pt}{0.400pt}}
\multiput(509.00,151.58)(0.543,0.492){19}{\rule{0.536pt}{0.118pt}}
\multiput(509.00,150.17)(10.887,11.000){2}{\rule{0.268pt}{0.400pt}}
\multiput(521.00,162.58)(0.497,0.493){23}{\rule{0.500pt}{0.119pt}}
\multiput(521.00,161.17)(11.962,13.000){2}{\rule{0.250pt}{0.400pt}}
\multiput(534.58,175.00)(0.493,0.576){23}{\rule{0.119pt}{0.562pt}}
\multiput(533.17,175.00)(13.000,13.834){2}{\rule{0.400pt}{0.281pt}}
\multiput(547.58,190.00)(0.493,0.655){23}{\rule{0.119pt}{0.623pt}}
\multiput(546.17,190.00)(13.000,15.707){2}{\rule{0.400pt}{0.312pt}}
\multiput(560.58,207.00)(0.492,0.755){21}{\rule{0.119pt}{0.700pt}}
\multiput(559.17,207.00)(12.000,16.547){2}{\rule{0.400pt}{0.350pt}}
\multiput(572.58,225.00)(0.493,0.774){23}{\rule{0.119pt}{0.715pt}}
\multiput(571.17,225.00)(13.000,18.515){2}{\rule{0.400pt}{0.358pt}}
\multiput(585.58,245.00)(0.493,0.814){23}{\rule{0.119pt}{0.746pt}}
\multiput(584.17,245.00)(13.000,19.451){2}{\rule{0.400pt}{0.373pt}}
\multiput(598.58,266.00)(0.492,0.970){21}{\rule{0.119pt}{0.867pt}}
\multiput(597.17,266.00)(12.000,21.201){2}{\rule{0.400pt}{0.433pt}}
\multiput(610.58,289.00)(0.493,0.933){23}{\rule{0.119pt}{0.838pt}}
\multiput(609.17,289.00)(13.000,22.260){2}{\rule{0.400pt}{0.419pt}}
\multiput(623.58,313.00)(0.493,0.933){23}{\rule{0.119pt}{0.838pt}}
\multiput(622.17,313.00)(13.000,22.260){2}{\rule{0.400pt}{0.419pt}}
\multiput(636.58,337.00)(0.492,1.056){21}{\rule{0.119pt}{0.933pt}}
\multiput(635.17,337.00)(12.000,23.063){2}{\rule{0.400pt}{0.467pt}}
\multiput(648.58,362.00)(0.493,0.972){23}{\rule{0.119pt}{0.869pt}}
\multiput(647.17,362.00)(13.000,23.196){2}{\rule{0.400pt}{0.435pt}}
\multiput(661.58,387.00)(0.493,0.972){23}{\rule{0.119pt}{0.869pt}}
\multiput(660.17,387.00)(13.000,23.196){2}{\rule{0.400pt}{0.435pt}}
\multiput(674.58,412.00)(0.493,0.972){23}{\rule{0.119pt}{0.869pt}}
\multiput(673.17,412.00)(13.000,23.196){2}{\rule{0.400pt}{0.435pt}}
\multiput(687.58,437.00)(0.492,1.013){21}{\rule{0.119pt}{0.900pt}}
\multiput(686.17,437.00)(12.000,22.132){2}{\rule{0.400pt}{0.450pt}}
\multiput(699.58,461.00)(0.493,0.893){23}{\rule{0.119pt}{0.808pt}}
\multiput(698.17,461.00)(13.000,21.324){2}{\rule{0.400pt}{0.404pt}}
\multiput(712.58,484.00)(0.493,0.814){23}{\rule{0.119pt}{0.746pt}}
\multiput(711.17,484.00)(13.000,19.451){2}{\rule{0.400pt}{0.373pt}}
\multiput(725.58,505.00)(0.492,0.841){21}{\rule{0.119pt}{0.767pt}}
\multiput(724.17,505.00)(12.000,18.409){2}{\rule{0.400pt}{0.383pt}}
\multiput(737.58,525.00)(0.493,0.695){23}{\rule{0.119pt}{0.654pt}}
\multiput(736.17,525.00)(13.000,16.643){2}{\rule{0.400pt}{0.327pt}}
\multiput(750.58,543.00)(0.493,0.616){23}{\rule{0.119pt}{0.592pt}}
\multiput(749.17,543.00)(13.000,14.771){2}{\rule{0.400pt}{0.296pt}}
\multiput(763.00,559.58)(0.539,0.492){21}{\rule{0.533pt}{0.119pt}}
\multiput(763.00,558.17)(11.893,12.000){2}{\rule{0.267pt}{0.400pt}}
\multiput(776.00,571.58)(0.600,0.491){17}{\rule{0.580pt}{0.118pt}}
\multiput(776.00,570.17)(10.796,10.000){2}{\rule{0.290pt}{0.400pt}}
\multiput(788.00,581.59)(0.950,0.485){11}{\rule{0.843pt}{0.117pt}}
\multiput(788.00,580.17)(11.251,7.000){2}{\rule{0.421pt}{0.400pt}}
\multiput(801.00,588.61)(2.695,0.447){3}{\rule{1.833pt}{0.108pt}}
\multiput(801.00,587.17)(9.195,3.000){2}{\rule{0.917pt}{0.400pt}}
\multiput(826.00,589.95)(2.695,-0.447){3}{\rule{1.833pt}{0.108pt}}
\multiput(826.00,590.17)(9.195,-3.000){2}{\rule{0.917pt}{0.400pt}}
\multiput(839.00,586.93)(0.950,-0.485){11}{\rule{0.843pt}{0.117pt}}
\multiput(839.00,587.17)(11.251,-7.000){2}{\rule{0.421pt}{0.400pt}}
\multiput(852.00,579.92)(0.600,-0.491){17}{\rule{0.580pt}{0.118pt}}
\multiput(852.00,580.17)(10.796,-10.000){2}{\rule{0.290pt}{0.400pt}}
\multiput(864.00,569.92)(0.539,-0.492){21}{\rule{0.533pt}{0.119pt}}
\multiput(864.00,570.17)(11.893,-12.000){2}{\rule{0.267pt}{0.400pt}}
\multiput(877.58,556.54)(0.493,-0.616){23}{\rule{0.119pt}{0.592pt}}
\multiput(876.17,557.77)(13.000,-14.771){2}{\rule{0.400pt}{0.296pt}}
\multiput(890.58,540.29)(0.493,-0.695){23}{\rule{0.119pt}{0.654pt}}
\multiput(889.17,541.64)(13.000,-16.643){2}{\rule{0.400pt}{0.327pt}}
\multiput(903.58,521.82)(0.492,-0.841){21}{\rule{0.119pt}{0.767pt}}
\multiput(902.17,523.41)(12.000,-18.409){2}{\rule{0.400pt}{0.383pt}}
\multiput(915.58,501.90)(0.493,-0.814){23}{\rule{0.119pt}{0.746pt}}
\multiput(914.17,503.45)(13.000,-19.451){2}{\rule{0.400pt}{0.373pt}}
\multiput(928.58,480.65)(0.493,-0.893){23}{\rule{0.119pt}{0.808pt}}
\multiput(927.17,482.32)(13.000,-21.324){2}{\rule{0.400pt}{0.404pt}}
\multiput(941.58,457.26)(0.492,-1.013){21}{\rule{0.119pt}{0.900pt}}
\multiput(940.17,459.13)(12.000,-22.132){2}{\rule{0.400pt}{0.450pt}}
\multiput(953.58,433.39)(0.493,-0.972){23}{\rule{0.119pt}{0.869pt}}
\multiput(952.17,435.20)(13.000,-23.196){2}{\rule{0.400pt}{0.435pt}}
\multiput(966.58,408.39)(0.493,-0.972){23}{\rule{0.119pt}{0.869pt}}
\multiput(965.17,410.20)(13.000,-23.196){2}{\rule{0.400pt}{0.435pt}}
\multiput(979.58,383.39)(0.493,-0.972){23}{\rule{0.119pt}{0.869pt}}
\multiput(978.17,385.20)(13.000,-23.196){2}{\rule{0.400pt}{0.435pt}}
\multiput(992.58,358.13)(0.492,-1.056){21}{\rule{0.119pt}{0.933pt}}
\multiput(991.17,360.06)(12.000,-23.063){2}{\rule{0.400pt}{0.467pt}}
\multiput(1004.58,333.52)(0.493,-0.933){23}{\rule{0.119pt}{0.838pt}}
\multiput(1003.17,335.26)(13.000,-22.260){2}{\rule{0.400pt}{0.419pt}}
\multiput(1017.58,309.52)(0.493,-0.933){23}{\rule{0.119pt}{0.838pt}}
\multiput(1016.17,311.26)(13.000,-22.260){2}{\rule{0.400pt}{0.419pt}}
\multiput(1030.58,285.40)(0.492,-0.970){21}{\rule{0.119pt}{0.867pt}}
\multiput(1029.17,287.20)(12.000,-21.201){2}{\rule{0.400pt}{0.433pt}}
\multiput(1042.58,262.90)(0.493,-0.814){23}{\rule{0.119pt}{0.746pt}}
\multiput(1041.17,264.45)(13.000,-19.451){2}{\rule{0.400pt}{0.373pt}}
\multiput(1055.58,242.03)(0.493,-0.774){23}{\rule{0.119pt}{0.715pt}}
\multiput(1054.17,243.52)(13.000,-18.515){2}{\rule{0.400pt}{0.358pt}}
\multiput(1068.58,222.09)(0.492,-0.755){21}{\rule{0.119pt}{0.700pt}}
\multiput(1067.17,223.55)(12.000,-16.547){2}{\rule{0.400pt}{0.350pt}}
\multiput(1080.58,204.41)(0.493,-0.655){23}{\rule{0.119pt}{0.623pt}}
\multiput(1079.17,205.71)(13.000,-15.707){2}{\rule{0.400pt}{0.312pt}}
\multiput(1093.58,187.67)(0.493,-0.576){23}{\rule{0.119pt}{0.562pt}}
\multiput(1092.17,188.83)(13.000,-13.834){2}{\rule{0.400pt}{0.281pt}}
\multiput(1106.00,173.92)(0.497,-0.493){23}{\rule{0.500pt}{0.119pt}}
\multiput(1106.00,174.17)(11.962,-13.000){2}{\rule{0.250pt}{0.400pt}}
\multiput(1119.00,160.92)(0.543,-0.492){19}{\rule{0.536pt}{0.118pt}}
\multiput(1119.00,161.17)(10.887,-11.000){2}{\rule{0.268pt}{0.400pt}}
\multiput(1131.00,149.93)(0.824,-0.488){13}{\rule{0.750pt}{0.117pt}}
\multiput(1131.00,150.17)(11.443,-8.000){2}{\rule{0.375pt}{0.400pt}}
\multiput(1144.00,141.93)(0.950,-0.485){11}{\rule{0.843pt}{0.117pt}}
\multiput(1144.00,142.17)(11.251,-7.000){2}{\rule{0.421pt}{0.400pt}}
\multiput(1157.00,134.94)(1.651,-0.468){5}{\rule{1.300pt}{0.113pt}}
\multiput(1157.00,135.17)(9.302,-4.000){2}{\rule{0.650pt}{0.400pt}}
\put(1169,130.67){\rule{3.132pt}{0.400pt}}
\multiput(1169.00,131.17)(6.500,-1.000){2}{\rule{1.566pt}{0.400pt}}
\put(1182,130.67){\rule{3.132pt}{0.400pt}}
\multiput(1182.00,130.17)(6.500,1.000){2}{\rule{1.566pt}{0.400pt}}
\multiput(1195.00,132.60)(1.797,0.468){5}{\rule{1.400pt}{0.113pt}}
\multiput(1195.00,131.17)(10.094,4.000){2}{\rule{0.700pt}{0.400pt}}
\multiput(1208.00,136.59)(1.033,0.482){9}{\rule{0.900pt}{0.116pt}}
\multiput(1208.00,135.17)(10.132,6.000){2}{\rule{0.450pt}{0.400pt}}
\multiput(1220.00,142.59)(0.728,0.489){15}{\rule{0.678pt}{0.118pt}}
\multiput(1220.00,141.17)(11.593,9.000){2}{\rule{0.339pt}{0.400pt}}
\multiput(1233.00,151.58)(0.539,0.492){21}{\rule{0.533pt}{0.119pt}}
\multiput(1233.00,150.17)(11.893,12.000){2}{\rule{0.267pt}{0.400pt}}
\multiput(1246.58,163.00)(0.492,0.625){21}{\rule{0.119pt}{0.600pt}}
\multiput(1245.17,163.00)(12.000,13.755){2}{\rule{0.400pt}{0.300pt}}
\multiput(1258.58,178.00)(0.493,0.655){23}{\rule{0.119pt}{0.623pt}}
\multiput(1257.17,178.00)(13.000,15.707){2}{\rule{0.400pt}{0.312pt}}
\multiput(1271.58,195.00)(0.493,0.774){23}{\rule{0.119pt}{0.715pt}}
\multiput(1270.17,195.00)(13.000,18.515){2}{\rule{0.400pt}{0.358pt}}
\multiput(1284.58,215.00)(0.493,0.893){23}{\rule{0.119pt}{0.808pt}}
\multiput(1283.17,215.00)(13.000,21.324){2}{\rule{0.400pt}{0.404pt}}
\multiput(1297.58,238.00)(0.492,1.142){21}{\rule{0.119pt}{1.000pt}}
\multiput(1296.17,238.00)(12.000,24.924){2}{\rule{0.400pt}{0.500pt}}
\multiput(1309.58,265.00)(0.493,1.131){23}{\rule{0.119pt}{0.992pt}}
\multiput(1308.17,265.00)(13.000,26.940){2}{\rule{0.400pt}{0.496pt}}
\multiput(1322.58,294.00)(0.493,1.250){23}{\rule{0.119pt}{1.085pt}}
\multiput(1321.17,294.00)(13.000,29.749){2}{\rule{0.400pt}{0.542pt}}
\multiput(1335.58,326.00)(0.492,1.487){21}{\rule{0.119pt}{1.267pt}}
\multiput(1334.17,326.00)(12.000,32.371){2}{\rule{0.400pt}{0.633pt}}
\multiput(1347.58,361.00)(0.493,1.488){23}{\rule{0.119pt}{1.269pt}}
\multiput(1346.17,361.00)(13.000,35.366){2}{\rule{0.400pt}{0.635pt}}
\multiput(1360.58,399.00)(0.493,1.646){23}{\rule{0.119pt}{1.392pt}}
\multiput(1359.17,399.00)(13.000,39.110){2}{\rule{0.400pt}{0.696pt}}
\multiput(1373.58,441.00)(0.492,1.875){21}{\rule{0.119pt}{1.567pt}}
\multiput(1372.17,441.00)(12.000,40.748){2}{\rule{0.400pt}{0.783pt}}
\multiput(1385.58,485.00)(0.493,1.845){23}{\rule{0.119pt}{1.546pt}}
\multiput(1384.17,485.00)(13.000,43.791){2}{\rule{0.400pt}{0.773pt}}
\multiput(1398.58,532.00)(0.493,2.003){23}{\rule{0.119pt}{1.669pt}}
\multiput(1397.17,532.00)(13.000,47.535){2}{\rule{0.400pt}{0.835pt}}
\multiput(1411.58,583.00)(0.493,2.122){23}{\rule{0.119pt}{1.762pt}}
\multiput(1410.17,583.00)(13.000,50.344){2}{\rule{0.400pt}{0.881pt}}
\multiput(1424.58,637.00)(0.492,2.392){21}{\rule{0.119pt}{1.967pt}}
\multiput(1423.17,637.00)(12.000,51.918){2}{\rule{0.400pt}{0.983pt}}
\multiput(1436.58,693.00)(0.493,2.360){23}{\rule{0.119pt}{1.946pt}}
\multiput(1435.17,693.00)(13.000,55.961){2}{\rule{0.400pt}{0.973pt}}
\put(814.0,591.0){\rule[-0.200pt]{2.891pt}{0.400pt}}
\put(191.0,131.0){\rule[-0.200pt]{0.400pt}{155.380pt}}
\put(191.0,131.0){\rule[-0.200pt]{303.052pt}{0.400pt}}
\put(1449.0,131.0){\rule[-0.200pt]{0.400pt}{155.380pt}}
\put(191.0,776.0){\rule[-0.200pt]{303.052pt}{0.400pt}}
\end{picture}
\end{center}\end{figure}
\subsection*{c)}
Die Extrema befinden sich an Stellen, an denen $F(x)=0$ gilt. Dies gilt bei $F_{Feder}=0$ und bei $x=0$, wo $F_{Feder}$ senkrecht auf $\vec x$ steht.
\begin{align}
F(x)&=0\\
F_{Feder}&=-k\left(\sqrt{L^2+x^2}-\ell_0\right)\\
\Leftrightarrow L^2+x^2&=\ell_0^2\\
\ell_0>L\Rightarrow x_{1,2}&=\pm\sqrt{\ell_0^2-L^2}
\end{align}
Für $\ell_0>L$ gibt es drei Gleichgewichtslagen: Eine bei $x=0$ und zwei bei $x=\pm\sqrt{\ell_0^2-L^2}$. Erstere ist instabil, das sie auf einem Potentialmaximum liegt (siehe Plot). Die anderen beiden sind stabil.

Für $\ell_0\leq L$ gibt es nur eine Gleichgewichtslage bei $x=0$, diese ist stabil.
\subsection*{d)}
Siehe obiger Paragraph zu b)
\subsection*{e)}

\aufgabe{2}
\begin{align}
\ddot x&=-\frac{\alpha}{mx^3}\\
V(x)&=-\int F(x)\diff x\\
&=\alpha\int\frac{1}{x^3}\diff x\\
&=-\alpha\frac{1}{2x^2}
\end{align}
\aufgabe{3}
\subsection*{a)}
\begin{align}
\frac{\partial f}{\partial t}&=2t+2x\\
\frac{\partial f}{\partial x}&=2t-\sin(x)
\end{align}
\subsection*{b)}
\begin{align}
f^{(t)}(x,t)&=\left(t^2+2ct^2+\cos\left(ct\right)\right)^{(t)}\\
&=(2+4c)t-c\cdot\sin\left(ct\right)\\
\frac{\diff f}{\diff t}&=\frac{\partial f}{\partial t}+\frac{\partial f}{\partial x}\cdot\frac{\diff x}{\diff t}\\
&=\left(2t+2ct\right)+\left(2t-\sin(ct)\right)\cdot c\\
&=\left(2+4c\right)t-c\cdot\sin(ct)
\end{align}
\aufgabe{4} %FIXME re-check.
\begin{align}
F(t)&=m_w(t)\cdot g\\
&=\frac{x}{\ell}mg\\
\ddot x&=\frac{F(t)}{m}\\
&=\frac{g}{\ell}x\\
x&=ae^{bt}\\
\left(ae^{bt}\right)''&=\left(abe^{bt}\right)'=ab^2e^{bt}\\
ab^2e^{bt}&=\frac{g}{\ell}ae^{bt}\\
\Leftrightarrow b^2&=\frac{g}{l}\\
\Leftrightarrow x&=ae^{\sqrt{\frac{g}{\ell}}t}\\
x(0)&=a=x_0\\
\Rightarrow x&=x_0e^{\sqrt{\frac{g}{\ell}}t}\\
E_{kin}&=\frac{1}{2}mv^2\\
&=\frac{1}{2}m\dot x^2\\
&=\frac{1}{2}mx_0^2\frac{g}{\ell}e^{2\sqrt{\frac{g}{\ell}}t}\\
&=\frac{mg}{2\ell}x_0^2e^{2\sqrt{\frac{g}{\ell}}t}\\
E_{pot}&=mgh-mg\frac{x}{\ell}h+mg\frac{x}{\ell}\left(h-\frac{x}{2}\right)\\
&=mg\cdot\left(h-\frac{x}{\ell}h+\frac{x}{\ell}h-\frac{x}{2}\frac{x}{\ell}\right)\\
&=mg\cdot\left(h-\frac{x^2}{2\ell}\right)\\
&=mgh-\frac{mg}{2\ell}x_0^2e^{2\sqrt{\frac{g}{\ell}}t}\\
E&=E_{kin}+E_{pot}\\
&=mgh
\end{align}
\end{document}
