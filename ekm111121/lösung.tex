\documentclass[12pt,a4paper,notitlepage]{article}
\usepackage[utf8x]{inputenc}
\usepackage[a4paper,textwidth=17cm, top=2cm, bottom=3.5cm]{geometry}
\usepackage{eurosym}
%\usepackage{url}
\usepackage[T1]{fontenc}
\usepackage{ucs}
\usepackage{ngerman} 
\usepackage{setspace}
%\usepackage{fourier}
\usepackage{amssymb,amsmath}
\usepackage{wasysym}
\usepackage{amsthm}
%\usepackage{marvosym}
\usepackage{tabularx}
\usepackage{multicol}
\usepackage{hyperref}
\usepackage{tabularx}
\usepackage[pdftex]{graphicx,color}
\usepackage{todo}
\definecolor{p-green}{rgb}{0.12,0.57,0.11}
\newcommand{\bitem}{\item[--]}
\newcommand{\litem}[2]{\item[#1 --] #2}
\newcommand{\blitem}[3]{\item[#1 --] \texttt{#2} -- #3}
\newcommand{\gfo}{\grqq\ }
\newcommand{\gfu}{\glqq}
\newcommand{\zquote}[2]{\glqq #1\grqq\ (Z.\ #2)}
\newcommand{\pquote}[1]{\glqq #1\grqq}
\newcommand{\nquote}[2]{#1: \glqq #2\grqq}
\newcommand{\nwquote}[3]{#1 -- \emph{#2}: \glqq #3\grqq}
\newcommand{\nwyquote}[4]{#1 -- \emph{#2} (#3): \glqq #4\grqq}
\newcommand{\diff}{\mathrm{d}}
\renewcommand{\abstractname}{}
\definecolor{orange}{rgb}{1,0.6,0}
\definecolor{d-green}{rgb}{0,0.8,0}
\definecolor{pink}{rgb}{1,0,0.6}
\newcommand{\annot}[1]{\textcolor{red}{#1}}
\newcommand{\grad}{\text{grad}}
\newcommand{\rot}{\text{rot}}
\newcommand{\ecolor}[1]{\textcolor{pink}{#1}}
\newcommand{\aufgabe}[1]{\section*{\setcounter{section}{#1}Aufgabe #1}}
\newcommand{\re}{\text{Re}}
\newcommand{\im}{\text{Im}}
\onehalfspacing
\setlength{\parskip}{8pt plus4pt minus4pt}
\begin{document}
\begin{center}
\Large Hausaufgaben für P1a

\normalsize Online unter \url{http://github.com/jaseg/Hausaufgaben}
\end{center}
\begin{tabularx}{\textwidth}{Xr}
Jan Sebastian Götte (546408), Paul Scheunemann&Abgabe: 111121
\end{tabularx}
\aufgabe{1}
\begin{align}
\dot v&=-m\gamma v+\frac{g_x}{m}\;;g_x=g\sin(x)\\
\text{Ansatz aus Wikipedia (en)}
\Rightarrow A(t)&=-m\gamma\\
b(t)&=\frac{g_x}{m}\\
F(t)&=\int_{t_0}^t-m\gamma\diff t\;;t_0=0\\
&=-m\gamma t\\
v(t)&=C(t)e^{-m\gamma t}\\
\dot v&=\dot Ce^{-m\gamma t}+\left(-m\gamma Ce^{-m\gamma t}\right)\\
&=-m\gamma v+\frac{g_x}{m}\\
\Rightarrow\frac{g_x}{m}=\dot Ce^{-m\gamma t}&\Leftrightarrow\frac{g_x}{m}e^{m\gamma t}=\dot C\\
C&=\int_{t_0}^te^{m\gamma t}\diff t\frac{g_x}{m}\\
\Rightarrow C&=\frac{g_x}{m}\cdot\frac{1}{m\gamma}e^{m\gamma t}+c\\
&=\frac{g_x}{\gamma m^2}e^{m\gamma t}+c\\
\Rightarrow v(t)&=\frac{g_x}{\gamma m^2}e^0+ce^{-m\gamma t}\\
\dot v&=-m\gamma ce^{-m\gamma t}\\
-m\gamma ce^{-m\gamma t}&=\frac{g_x}{m}-m\gamma ce^{-m\gamma t}+\frac{g_x}{m}\\
v(0)=0&=\frac{g_x}{\gamma m^2}e^0+ce^{-m\gamma t}\\
\Rightarrow c=-\frac{g_x}{\gamma m^2}\\
x(t)&=\int_{t_0}^tv(t)\diff t\\
&=\int \frac{g_x}{\gamma m^2}+\frac{g_x}{\gamma m^2}e^{-m\gamma t}\diff t\\
&=t\frac{g_x}{\gamma m^2}-\frac{g_x}{m}e^{-m\gamma t}+c_2\\
x(0)=0&=-\frac{g_x}{m}+c_2\\
\Rightarrow c_2&=\frac{g_x}{m}\\
x(t)&=t\frac{g_x}{\gamma m^2}-\frac{g_x}{m}e^{-m\gamma t}+\frac{g_x}{m}
\end{align}
\aufgabe{2}
\aufgabe{3}
\begin{align}
\grad\left(\frac{a}{r}\right)&=-\frac{a}{r^2}\\
\rot(\vec\omega\times\vec r)&=\nabla\times(\vec\omega\times\vec r)\\
&=\nabla\times\left(\begin{matrix}
\omega_2r_3-\omega_3r_2\\
\omega_3r_1-\omega_1r_3\\
\omega_1r_2-\omega_2r_1
\end{matrix}\right)\\
&=\left(\begin{matrix}
-\omega_1-\omega_1\\
-\omega_2-\omega_2\\
-\omega_3-\omega_3
\end{matrix}\right)\\
&=-2\vec\omega\\
\rot\left(\begin{matrix}
e^{-x^2-y^2}\\
e^{-x^2-y^2}\\
z
\end{matrix}\right)&=\left(\begin{matrix}
0\\
0\\
-2ye^{-x^2-y^2}+2xe^{-x^2-y^2}
\end{matrix}\right)\\
&=\left(\begin{matrix}
0\\
0\\
(2x-2y)e^{-x^2-y^2}
\end{matrix}\right)\\
\rot\left(\begin{matrix}
xe^{\sin(r)}\\
ye^{\sin(r)}\\
ze^{\sin(r)}
\end{matrix}\right)&;\;r=\sqrt{x^2+y^2+z^2}\\
&=\vec 0\;\text{, da gilt: }\\
\left(ze^{\sin(r)}\right)^{(y)}&=yz\frac{e^{\sin(r)}\cos(r)}{r}
\end{align}
\end{document}
