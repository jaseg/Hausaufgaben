\documentclass[12pt,a4paper,notitlepage]{article}
\usepackage[utf8x]{inputenc}
\usepackage[a4paper,textwidth=17cm, top=2cm, bottom=3.5cm]{geometry}
\usepackage{eurosym}
%\usepackage{url}
\usepackage[T1]{fontenc}
\usepackage{ucs}
\usepackage{ngerman} 
\usepackage{setspace}
%\usepackage{fourier}
\usepackage{amssymb,amsmath}
\usepackage{wasysym}
\usepackage{amsthm}
%\usepackage{marvosym}
\usepackage{tabularx}
\usepackage{multicol}
\usepackage{hyperref}
\usepackage[pdftex]{graphicx,color}
\usepackage{todo}
\definecolor{p-green}{rgb}{0.12,0.57,0.11}
\newcommand{\bitem}{\item[--]}
\newcommand{\litem}[2]{\item[#1 --] #2}
\newcommand{\blitem}[3]{\item[#1 --] \texttt{#2} -- #3}
\newcommand{\gfo}{\grqq\ }
\newcommand{\gfu}{\glqq}
\newcommand{\zquote}[2]{\glqq #1\grqq\ (Z.\ #2)}
\newcommand{\pquote}[1]{\glqq #1\grqq}
\newcommand{\nquote}[2]{#1: \glqq #2\grqq}
\newcommand{\nwquote}[3]{#1 -- \emph{#2}: \glqq #3\grqq}
\newcommand{\nwyquote}[4]{#1 -- \emph{#2} (#3): \glqq #4\grqq}
\newcommand{\diff}{\mathrm{d}}
\renewcommand{\abstractname}{}
\definecolor{orange}{rgb}{1,0.6,0}
\definecolor{d-green}{rgb}{0,0.8,0}
\definecolor{pink}{rgb}{1,0,0.6}
\newcommand{\annot}[1]{\textcolor{red}{#1}}
\newcommand{\ecolor}[1]{\textcolor{pink}{#1}}
\newcommand{\aufgabe}[1]{\section*{\setcounter{section}{#1}Aufgabe #1}}
\newcommand{\re}{\text{Re}}
\newcommand{\im}{\text{Im}}
\onehalfspacing
\setlength{\parskip}{8pt plus4pt minus4pt}
\title{}
\begin{document}
\aufgabe{1}
\subsection*{a)}
\begin{equation}
n\in\mathbb{N}\Leftrightarrow \sum^n_{j=1}j^3=\frac{1}{4}n^2(n+1)^2
\end{equation}
\subsection*{b)}$2^4\ngtr5^2, 2^5>5^2$, $n^a$ wächst schneller als $a^n$. Also gilt:
\begin{equation}
n\in\mathbb{N}; n>4\Leftrightarrow 2^n>n^2
\end{equation}
\subsection*{c)}$2^{2+1}\ngtr(2+1)^2, 3^{3+1}>(3+1)^3$, $n^{n+1}$ wächst schneller als $(n+1)^n$. Also gilt:
\begin{equation}
n\in\mathbb{N}; n>2\Leftrightarrow n^{n+1}>(n+1)^n
\end{equation}
\aufgabe{2}
\subsection*{a)}
Wahr.
\subsection*{b)}
Unwahr. Ein $(x\in A,y\in D)$ wäre nur in der linken Menge enthalten.
\aufgabe{4}
\begin{description}
\item[Reflexivität] Per Definition gegeben, da $mn=mn$.
\item[Symmetrie] Gegeben, da $(m_1,n_1)\sim(m_2,n_2)\Leftrightarrow(m_2,n_2)\sim(m_1,n_1)$ heißt, dass $m_1n_2=m_2n_1\Leftrightarrow m_2n_1=m_1n_2$, was offenbar wahr ist.
\item[Transitivität]
\begin{align}
m_1n_2=m_2n_1&,m_2n_3=m_3n_2\;\left|\begin{array}{l}\text{Beidseitige Multiplikation der}\\\text{rechten mit der linken Gleichung}\end{array}\right.\\
\Leftrightarrow m_1n_2m_2n_3&=m_2n_1m_3n_2\;\Big|\text{Beidseitiges Teilen durch }m_2n_2\\
\Leftrightarrow m_1n_3&=m_3n_1\qed
\end{align}
\end{description}
Folglich ist $\sim$ eine Äquivalenzrelation.
\begin{align}
M&=\mathbb Z\times\mathbb N\\
[(1,1)]_\sim&=\left\{a\in\mathbb N:(a,a)\right\}\subseteq M\\
[(1,2)]_\sim&=\left\{a\in\mathbb N:(2a,a)\right\}
\end{align}
\aufgabe{0}
Ja. z.B.:
\begin{align}
A&=\left[0,1\right]\\
B&=\left]0,\right[\\
A\rightarrow B:=b(a)&=\left\{\begin{array}{cl}\frac{1}{\frac{1}{a}+1}&\text{ falls }\frac{1}{a}\in\mathbb{N}\\a&\text{ sonst.}\end{array}\right.\\
B\rightarrow A:=a(b)&=\left\{\begin{array}{cl}\frac{1}{\frac{1}{b}-1}&\text{ falls }\frac{1}{b}\in\mathbb{N}\\b&\text{ sonst.}\end{array}\right.
\end{align}
\end{document}
