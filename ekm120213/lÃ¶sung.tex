\documentclass[12pt,a4paper,notitlepage]{article}
\usepackage[utf8]{inputenc}
\usepackage[a4paper,textwidth=17cm, top=2cm, bottom=3.5cm]{geometry}
%\usepackage{url}
\usepackage[T1]{fontenc}
\usepackage{ucs}
\usepackage{ngerman} 
\usepackage{setspace}
%\usepackage{fourier}
\usepackage{amssymb,amsmath}
\usepackage{eurosym}
\usepackage{wasysym}
\usepackage{spverbatim}
%\usepackage{gensymb}
\usepackage{amsthm}
%\usepackage{marvosym}
\usepackage{tabularx}
\usepackage{multicol}
\usepackage{hyperref}
\usepackage[thinspace,thinqspace,squaren,textstyle]{SIunits}
\usepackage{tabularx}
\usepackage[pdftex]{graphicx,color}
%\usepackage{todo}
\definecolor{p-green}{rgb}{0.12,0.57,0.11}
\newcommand{\bitem}{\item[--]}
\newcommand{\litem}[2]{\item[#1 --] #2}
\newcommand{\blitem}[3]{\item[#1 --] \texttt{#2} -- #3}
\newcommand{\gfo}{\grqq\ }
\newcommand{\gfu}{\glqq}
\newcommand{\zquote}[2]{\glqq #1\grqq\ (Z.\ #2)}
\newcommand{\pquote}[1]{\glqq #1\grqq}
\newcommand{\nquote}[2]{#1: \glqq #2\grqq}
\newcommand{\nwquote}[3]{#1 -- \emph{#2}: \glqq #3\grqq}
\newcommand{\nwyquote}[4]{#1 -- \emph{#2} (#3): \glqq #4\grqq}
\newcommand{\diff}{\mathrm{d}}
\renewcommand{\abstractname}{}
\definecolor{orange}{rgb}{1,0.6,0}
\definecolor{d-green}{rgb}{0,0.8,0}
\definecolor{pink}{rgb}{1,0,0.6}
\newcommand{\annot}[1]{\textcolor{red}{#1}}
\newcommand{\ecolor}[1]{\textcolor{pink}{#1}}
\newcommand{\aufgabe}[1]{\section*{\setcounter{section}{#1}Aufgabe #1}}
\newcommand{\re}{\text{Re}}
\newcommand{\im}{\text{Im}}
\onehalfspacing
\setlength{\parskip}{8pt plus4pt minus4pt}
\begin{document}
\begin{center}
\Large Hausaufgaben für P1a

\normalsize Online unter \url{http://github.com/jaseg/Hausaufgaben}
\end{center}
\begin{tabularx}{\textwidth}{Xr}
Jan Sebastian Götte (546408)
\end{tabularx}
\aufgabe{1}
\begin{align}
\rho_L &= \frac{1}{V_m}\cdot\sum M_i\cdot r_i\\
V_m \propto T &\Rightarrow V_m(T=20\degree) = V_m(T=0\degree)\cdot\frac{293.15\kelvin}{273.15\kelvin}\\
\Rightarrow \rho_L &= \frac{1}{V_m(T=0\degree)\cdot\frac{293.15\kelvin}{273.15\kelvin}}\cdot\sum M_i\cdot r_i\\
&= 1.204041\frac{\kilogram}{\meter^3}
\end{align}
\aufgabe{2}
\subsection*{a)}
\begin{align}
\rho(T_2)&=\rho(T_1)\cdot\frac{T_1}{T_2}\\
F_A=g\rho V&\;F_g=g\rho V=mg\\
300\kilogram\cdot g&=g\rho_LV-g\rho_HV\\
\Rightarrow 300\kilogram&=(\rho_L-\rho_H)\cdot V\\
&=1.2\kilogram/\meter^3\cdot\left(1-\frac{T_1}{T_2}\right)\cdot V\\
V=\frac{\pi}{6}d^3\\
\Rightarrow d&=\sqrt[3]{\frac{6\cdot 300}{\pi\cdot 1.2\cdot\left(1-\frac{393.15}{413.15}\right)}}\\
&=21.5\meter
\end{align}
\subsection*{b)}
Die Dichte von Wasserdampf liegt bei ca. $0.6\frac{\kilogram}{\meter^3}$, was geringer als die Dichte 40° warmer Luft ist. Der Ballon wird also leichter und steigt somit.
\aufgabe{3}
\begin{align}
pV&=nRT\\
n&=\frac{pV}{RT}\\
p&=\frac{nRT}{V}\\
n_\text{ges}&=\sum_i\frac{p_iV_i}{RT_i}\\
p_\text{ges}&=\frac{n_\text{ges}RT_\text{ges}}{V}\\
U_\text{ges}&=U_1+U_2=\frac{5}{2}R\left(n_1T_1+n_2T_2\right)\\
&=n_\text{ges}T_\text{ges}\cdot\frac{5}{2}R\\
\Rightarrow T_\text{ges}&=\frac{n_1T_1+n_2T_2}{n_\text{ges}}\\
&=\frac{\frac{T_1p_1V_1}{RT_1}+\frac{T_2p_2V_2}{RT_2}}{\frac{p_1V_1}{RT_1}+\frac{p_2V_2}{RT_2}}\\
&=\frac{p_1V_1+p_2V_2}{\frac{p_1V_1}{T_1}+\frac{p_2V_2}{T_2}}\\
&=\frac{\frac{1}{3}p_1+\frac{2}{3}p_2}{\frac{p_1}{3T_1}+\frac{2p_2}{3T_2}}\\
&=336K\\
\Rightarrow p_\text{ges}&=\frac{R\left(n_1T_1+n_2T_2\right)}{V}\\
&=\frac{R}{V}\left(\frac{p_1V_1T_1}{RT_1}+\frac{p_2V_2T_2}{RT_2}\right)\\
&=\frac{p_1V_1+p_2V_2}{V}\\
&=\frac{p_1\cdot\frac{1}{3}V+p_2\cdot\frac{2}{3}V}{V}\\
&=\frac{1}{3}p_1+\frac{2}{3}p_2\\
&=1.2\cdot 10^6\pascal
\end{align}
\aufgabe{4}
\aufgabe{5}
\subsection*{a)}
\begin{equation}
\rho_X(x)=\left\{\begin{matrix}\frac{1}{b-a}&\;\text{für}\;a\leq x\leq b\\0&\;\text{sonst}\end{matrix}\right.
\end{equation}
\subsection*{b)}
\begin{equation}
\left<x\right>=\int x\rho_X(x)\diff x=\frac{x}{2(b-a)}\big|_a^b=\frac{b^2-a^2}{2(b-a)}=\frac{(b+a)(b-a)}{2(b-a)}=\frac{a+b}{2}
\end{equation}
\subsection*{c)}
\begin{align}
\sigma_X&=\sqrt{\mathrm{Var}(X)}=\sqrt{\int_{-\infty}^{\infty} (x-\left<x\right>)^2\rho_X(x)\diff x}
\end{align}
\end{document}
