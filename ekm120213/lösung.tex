\documentclass[12pt,a4paper,notitlepage]{article}
\usepackage[utf8]{inputenc}
\usepackage[a4paper,textwidth=17cm, top=2cm, bottom=3.5cm]{geometry}
%\usepackage{url}
\usepackage[T1]{fontenc}
\usepackage{ucs}
\usepackage{ngerman} 
\usepackage{setspace}
\usepackage{fourier}
\usepackage{amssymb,amsmath}
\usepackage{eurosym}
\usepackage{wasysym}
\usepackage{spverbatim}
%\usepackage{gensymb}
\usepackage{amsthm}
%\usepackage{marvosym}
\usepackage{tabularx}
\usepackage{multicol}
\usepackage{hyperref}
\usepackage[thinspace,thinqspace,squaren,textstyle]{SIunits}
\usepackage{tabularx}
\usepackage[pdftex]{graphicx,color}
%\usepackage{todo}
\definecolor{p-green}{rgb}{0.12,0.57,0.11}
\newcommand{\bitem}{\item[--]}
\newcommand{\litem}[2]{\item[#1 --] #2}
\newcommand{\blitem}[3]{\item[#1 --] \texttt{#2} -- #3}
\newcommand{\gfo}{\grqq\ }
\newcommand{\gfu}{\glqq}
\newcommand{\zquote}[2]{\glqq #1\grqq\ (Z.\ #2)}
\newcommand{\pquote}[1]{\glqq #1\grqq}
\newcommand{\nquote}[2]{#1: \glqq #2\grqq}
\newcommand{\nwquote}[3]{#1 -- \emph{#2}: \glqq #3\grqq}
\newcommand{\nwyquote}[4]{#1 -- \emph{#2} (#3): \glqq #4\grqq}
\newcommand{\diff}{\mathrm{d}}
\renewcommand{\abstractname}{}
\definecolor{orange}{rgb}{1,0.6,0}
\definecolor{d-green}{rgb}{0,0.8,0}
\definecolor{pink}{rgb}{1,0,0.6}
\newcommand{\annot}[1]{\textcolor{red}{#1}}
\newcommand{\ecolor}[1]{\textcolor{pink}{#1}}
\newcommand{\aufgabe}[1]{\section*{\setcounter{section}{#1}Aufgabe #1}}
\newcommand{\re}{\text{Re}}
\newcommand{\im}{\text{Im}}
\onehalfspacing
\setlength{\parskip}{8pt plus4pt minus4pt}
\begin{document}
\begin{center}
\Large Hausaufgaben für P1a

\normalsize Online unter \url{http://github.com/jaseg/Hausaufgaben}
\end{center}
\begin{tabularx}{\textwidth}{Xr}
Jan Sebastian Götte (546408)
\end{tabularx}
\aufgabe{1}
\begin{align}
\rho_L &= \frac{1}{V_m}\cdot\sum M_i\cdot r_i\\
V_m \propto T &\Rightarrow V_m(T=20\degree) = V_m(T=0\degree)\cdot\frac{293.15\kelvin}{273.15\kelvin}\\
\Rightarrow \rho_L &= \frac{1}{V_m(T=0\degree)\cdot\frac{293.15\kelvin}{273.15\kelvin}}\cdot\sum M_i\cdot r_i\\
&= 1.204041\frac{\kilogram}{\meter^3}
\end{align}
\aufgabe{2}
\subsection*{a)}
\begin{align}
\rho(T_2)&=\rho(T_1)\cdot\frac{T_1}{T_2}\\
F_A=g\rho V&\;F_g=g\rho V=mg\\
300\kilogram\cdot g&=g\rho_LV-g\rho_HV\\
\Rightarrow 300\kilogram&=(\rho_L-\rho_H)\cdot V\\
&=1.2\kilogram/\meter^3\cdot\left(1-\frac{T_1}{T_2}\right)\cdot V\\
V=\frac{\pi}{6}d^3\\
\Rightarrow d&=\sqrt[3]{\frac{6\cdot 300}{\pi\cdot 1.2\cdot\left(1-\frac{393.15}{413.15}\right)}}\\
&=21.5\meter
\end{align}
\subsection*{b)}
Die Dichte von Wasserdampf liegt bei ca. $0.6\frac{\kilogram}{\meter^3}$, was geringer als die Dichte 40° warmer Luft ist. Der Ballon wird also leichter und steigt somit.
\aufgabe{3}
Notiz: Das hier ist alles irgendwie falsch, da Argon und Helium Edelgase (und damit einatomig) sind und nur 3 freiheitsgrade besitzen.
\begin{align}
pV&=nRT\\
n&=\frac{pV}{RT}\\
p&=\frac{nRT}{V}\\
n_\text{ges}&=\sum_i\frac{p_iV_i}{RT_i}\\
p_\text{ges}&=\frac{n_\text{ges}RT_\text{ges}}{V}\\
U_\text{ges}&=U_1+U_2=\frac{5}{2}R\left(n_1T_1+n_2T_2\right)\\
&=n_\text{ges}T_\text{ges}\cdot\frac{5}{2}R\\
\Rightarrow T_\text{ges}&=\frac{n_1T_1+n_2T_2}{n_\text{ges}}\\
&=\frac{\frac{T_1p_1V_1}{RT_1}+\frac{T_2p_2V_2}{RT_2}}{\frac{p_1V_1}{RT_1}+\frac{p_2V_2}{RT_2}}\\
&=\frac{p_1V_1+p_2V_2}{\frac{p_1V_1}{T_1}+\frac{p_2V_2}{T_2}}\\
&=\frac{\frac{1}{3}p_1+\frac{2}{3}p_2}{\frac{p_1}{3T_1}+\frac{2p_2}{3T_2}}\\
&=336K\\
\Rightarrow p_\text{ges}&=\frac{R\left(n_1T_1+n_2T_2\right)}{V}\\
&=\frac{R}{V}\left(\frac{p_1V_1T_1}{RT_1}+\frac{p_2V_2T_2}{RT_2}\right)\\
&=\frac{p_1V_1+p_2V_2}{V}\\
&=\frac{p_1\cdot\frac{1}{3}V+p_2\cdot\frac{2}{3}V}{V}\\
&=\frac{1}{3}p_1+\frac{2}{3}p_2\\
&=1.2\cdot 10^6\pascal
\end{align}
\subsection*{Neuer Ansatz unter Berücksichtigung der unterschiedlichen Freiheitsgrade:}
\begin{align}
pV&=nRT\\
n&=\frac{pV}{RT}\\
p&=\frac{nRT}{V}\\
n_\text{ges}&=\sum_i\frac{p_iV_i}{RT_i}\\
p_\text{ges}&=\frac{n_\text{ges}RT_\text{ges}}{V}\\
U_\text{ges}&=U_1+U_2=\frac{R}{2}\left(f_1n_1T_1+f_2n_2T_2\right)\\
&=n_\text{ges}T_\text{ges}\cdot\frac{f_\text{ges}}{2}R\\
\Rightarrow T_\text{ges}&=\left(\frac{f_1}{f_\text{ges}}n_1T_1+\frac{f_2}{f_\text{ges}}n_2T_2\right)\cdot\frac{1}{n_\text{ges}}\\
\text{Vermutung:}\quad f_\text{ges}&=\frac{f_1n_1}{n_\text{ges}}+\frac{f_2n_2}{n_\text{ges}}\\
\text{Damit ergibt sich: }T_\text{ges}&=\left(\frac{f_1n_\text{ges}}{f_1n_1+f_2n_2}n_1T_1+\frac{f_2n_\text{ges}}{f_1n_1+f_2n_2}n_2T_2\right)\cdot\frac{1}{n_\text{ges}}\\
&=\frac{n_1T_1}{n_1+\frac{f_2}{f_1}n_2}+\frac{n_2T_2}{\frac{f_1}{f_2}n_1+n_2}\\
&=\frac{T_1}{1+\frac{f_2n_2}{f_1n_1}}+\frac{T_2}{1+\frac{f_1n_1}{f_2n_2}}\\
\text{Ergebnis aus Maixma: }T_\text{ges}&=\\
%begin uncorrected
\Rightarrow p_\text{ges}&=\frac{R\left(n_1T_1+n_2T_2\right)}{V}
\end{align}
Hier hat sich nix geändert... kommt mir auch irgendwie komisch vor.
\begin{align}
&=\frac{R}{V}\left(\frac{p_1V_1T_1}{RT_1}+\frac{p_2V_2T_2}{RT_2}\right)\\
&=\frac{p_1V_1+p_2V_2}{V}\\
&=\frac{p_1\cdot\frac{1}{3}V+p_2\cdot\frac{2}{3}V}{V}\\
&=\frac{1}{3}p_1+\frac{2}{3}p_2\\
&=1.2\cdot 10^6\pascal
\end{align}
\aufgabe{4}
\subsection*{a)}
\begin{align}
F(V)&=4\pi\left(\frac{M}{2\pi RT}\right)^{\frac{3}{2}}v^2\exp\left(-\frac{MV^2}{2RT}\right)\\
k&:=\frac{M}{2RT}\\
\Rightarrow F(v)&=4\pi\left(\frac{k}{\i}\right)^\frac{3}{2}v^2\exp\left(-kv^2\right)\\
0=\frac{\diff F}{\diff v}&=8\pi\left(\frac{k}{\pi}\right)^\frac{3}{2}v\exp\left(-kv^2\right)-8\pi k\left(\frac{k}{\pi}\right)^\frac{3}{2}v^3\exp\left(-kv^2\right)\\
&=\left(v-kv^3\right)\underbrace{\left(8\pi\left(\frac{k}{\pi}\right)^\frac{3}{2}\right)}_{\neq 0}\underbrace{\exp\left(-kv^2\right)}_{>0\forall v}\\
\Rightarrow v&=kv^3\\
\Rightarrow v&=\sqrt{\frac{1}{k}}\;\text{da}\;v>0\\
\Rightarrow v&=\sqrt{\frac{2RT}{M}}
\end{align}
Ist vielleicht der eindimensionale Fall gemeint?
\begin{align}
F(v)&=\sqrt{\frac{m_M}{2\pi k_BT}}\exp\left(-\frac{m_Mv^2}{2k_BT}\right)\label{mb1dim}\\
\mathrm{Var}(F)&=\frac{m_M}{2\pi k_BT}\mathrm{Var}\left(\exp\left(-\frac{m_Mv^2}{2k_BT}\right)\right)
\end{align}
Für eine Normalverteilung gilt:
\begin{align}
f(x)&=\frac{1}{\sqrt{2}\sigma}\exp\left(-\frac{1}{2}\left(\frac{x-\mu}{\sigma}\right)^2\right)
\end{align}
Demzufolge ist die Varianz der in Gl.\ \ref{mb1dim} beschriebenen gleich der einer Normalverteilung, deren Exponent gleich dem Exponent der Maxwell-Boltzmann-Verteilung ist. Sie kann nun durch Gleichsetzen der Exponenten bestimmt werden ($\mu$ ist offenbar 0):
\begin{align}
-\frac{1}{2\sigma^2}\cdot v^2&=-\frac{m_M}{2k_BT}\cdot v^2\\
\Rightarrow \frac{1}{\sigma^2}&=\frac{m_M}{k_BT}\\
\Rightarrow \sigma&=\sqrt{\frac{k_BT}{m_M}}
\end{align}
Falls die mehrdimensionale Maxwell-Boltzmann-Geschwindigkeitsverteilung gemeint ist: Die Stanardabweichung einer Maxwell-Verteilung mit Parameter $a=\sqrt{\frac{RT}{M}}$ ist lt.\ Wikipedia
\begin{align}
\sigma_V&=a\sqrt{\frac{3\pi-8}{\pi}}=\sqrt{\frac{RT}{M}\cdot\left(3-\frac{8}{\pi}\right)}
\end{align}
Eine Herleitung mit Maxima findet sich im Maxima-Log im Anhang.
\subsection*{b)}
Da es ziemlich müßig ist, die kumulative Verteilungsfunktion händisch herzuleiten, habe ich meinen Computer gefragt, und der hat mir erzählt, dass diese für die Maxwell-Verteilung
\begin{align}
F(v)&=\sqrt{\frac{2}{\pi}}\frac{x^2}{a^3}\exp\left(\frac{-x^2}{2a^2}\right)\;\text{mit}\;a=\sqrt{\frac{k_BT}{m_M}}=\sqrt{\frac{RT}{M}}\\
D_V(v)&=\mathrm{erf}\left(\frac{v}{\sqrt{2}a}\right)-\sqrt{\frac{2}{\pi}}\frac{v}{a}\exp\left(\frac{-v^2}{2a^2}\right)\\
\end{align}
ist, was für die Wahrscheinlichkeit, ein Stickstoffmolekül im Intervall $\left[400,500\right]\frac{\meter}{\second}$ anzutreffen folgendes ergibt:
\begin{align}
\Pr_{N_2,T=300\kelvin}(v\mid v\in\left[400,500\right]\frac{\meter}{\second})&=D_{V(M,T)}(500)-D_{V(M,T)}(400)
\end{align}
\begin{align}
&=\mathrm{erf}\left(\frac{500\meter}{\sqrt{2}a\second}\right)-\sqrt{\frac{2}{\pi}}\frac{500\meter}{a\second}\exp\left(\frac{-500^2\meter^2}{2a^2\second^2}\right)-\mathrm{erf}\left(\frac{400\meter}{\sqrt{2}a\second}\right)-\sqrt{\frac{2}{\pi}}\frac{400\meter}{a\second}\exp\left(\frac{-400^2\meter^2}{2a^2\second^2}\right)\\
\left[a\right]&=\sqrt{\frac{\kilogram\cdot\meter^2\cdot\kelvin\cdot\mole\cdot\second^2}{\kelvin\cdot\mole\cdot\kilogram}}=\frac{\meter}{\second}\\
a&=\sqrt{\frac{R\cdot 300\kelvin\cdot\mole}{0.0280134\kilogram}}
\end{align}
\begin{align}
\Pr_{N_2,T=300\kelvin}(v\mid v\in\left[400,500\right]\frac{\meter}{\second})&=0.376
\end{align}
Der genaue Rechenweg steht im angehängten maxima-log.
Die mittlere Geschwindigkeit ist $\mu$ der o.\;g.\ Verteilung, lt.\ Wikipedia (Englisch) ist das
\begin{align}
\mu_V&=2a\sqrt{\frac{2}{\pi}}=476\frac{\meter}{\second}
\end{align}
\aufgabe{5}
\subsection*{a)}
\begin{equation}
\rho_X(x)=\left\{\begin{matrix}\frac{1}{b-a}&\;\text{für}\;a\leq x\leq b\\0&\;\text{sonst}\end{matrix}\right.
\end{equation}
\subsection*{b)}
\begin{equation}
\left<x\right>=\int x\rho_X(x)\diff x=\frac{x}{2(b-a)}\big|_a^b=\frac{b^2-a^2}{2(b-a)}=\frac{(b+a)(b-a)}{2(b-a)}=\frac{a+b}{2}
\end{equation}
\subsection*{c)}
\begin{align}
\sigma_X&=\sqrt{\mathrm{Var}(X)}\\
\mathrm{Var}(X)&=\left<X^2\right>-\left<X\right>^2\\
&=\frac{1}{b-a}\int_a^bx^2\diff x-\frac{\left(a+b\right)^2}{4}\\
&=\frac{1}{b-a}\frac{b^3-a^3}{3}-\frac{\left(a+b\right)^2}{4}\\
&=\frac{1}{3}\left(a^2+ab+b^2\right)-\frac{\left(a+b\right)^2}{4}\\
&=\frac{1}{12}\left(4a^2+4ab+4b^2-3a^2-6ab-3b^2\right)\\
&=\frac{1}{12}\left(a^2-2ab+b^2\right)\\
&=\frac{\left(b-a\right)^2}{12}\\
\Rightarrow \sigma_X&=\frac{b-a}{2\sqrt{3}}
\end{align}
\end{document}
