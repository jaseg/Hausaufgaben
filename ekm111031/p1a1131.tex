\documentclass[12pt,a4paper,notitlepage]{article}
\usepackage[utf8x]{inputenc}
\usepackage[a4paper,textwidth=17cm, top=2cm, bottom=3.5cm]{geometry}
\usepackage{eurosym}
%\usepackage{url}
\usepackage[T1]{fontenc}
\usepackage{ucs}
\usepackage{ngerman} 
\usepackage{setspace}
%\usepackage{fourier}
\usepackage{amssymb,amsmath}
\usepackage{wasysym}
%\usepackage{marvosym}
\usepackage{tabularx}
\usepackage{multicol}
\usepackage{hyperref}
\usepackage[pdftex]{graphicx,color}
\usepackage{todo}
\usepackage{amsthm}
\definecolor{p-green}{rgb}{0.12,0.57,0.11}
\newcommand{\bitem}{\item[--]}
\newcommand{\litem}[2]{\item[#1 --] #2}
\newcommand{\blitem}[3]{\item[#1 --] \texttt{#2} -- #3}
\newcommand{\gfo}{\grqq\ }
\newcommand{\gfu}{\glqq}
\newcommand{\zquote}[2]{\glqq #1\grqq\ (Z.\ #2)}
\newcommand{\pquote}[1]{\glqq #1\grqq}
\newcommand{\nquote}[2]{#1: \glqq #2\grqq}
\newcommand{\nwquote}[3]{#1 -- \emph{#2}: \glqq #3\grqq}
\newcommand{\nwyquote}[4]{#1 -- \emph{#2} (#3): \glqq #4\grqq}
\newcommand{\diff}{\mathrm{d}}
\renewcommand{\abstractname}{}
\definecolor{orange}{rgb}{1,0.6,0}
\definecolor{d-green}{rgb}{0,0.8,0}
\definecolor{pink}{rgb}{1,0,0.6}
\newcommand{\annot}[1]{\textcolor{red}{#1}}
\newcommand{\ecolor}[1]{\textcolor{pink}{#1}}
\newcommand{\aufgabe}[1]{\section*{\setcounter{section}{#1}Aufgabe #1}}
\newcommand{\re}{\text{Re}}
\newcommand{\im}{\text{Im}}
\onehalfspacing
\numberwithin{equation}{section}
\setlength{\parskip}{8pt plus4pt minus4pt}
\title{}
\begin{document}
\aufgabe{1}
\begin{align}
h'(u)&=\frac{u}{\sqrt{u^2+x^2}}\\
F'(y)&=\frac{-\ln a}{a}\\
F'(a)&=-ya^{-y-1}
\end{align}
\aufgabe{2}
\begin{align}
I_1&=\frac{1}{2}\int_0^{2\pi}sin(x)\diff x\\
&=0\\
I_2&=\left.\sin x-x\cos x\right|_0^\pi
\end{align}
\aufgabe{3}
\begin{align}
&10^{-3}m\\
&7.6\cdot10^{-7}m^2\\
&10^{-3}m^3\\
&0.7\cdot 10^{-9}kg\\
&86400s\\
&31557600s\\
&10\frac{m}{s}\\
&7.7\frac{m}{s^2}
\end{align}
\aufgabe{4}
\begin{align}
\overline V&=\overline \ell^3\\
&=(200mm)^3\\
&=(2.00\cdot10^{-1})^3m^3\\
&=8.00\cdot10^{-3}m^3\qed\\
\overline V+\Delta V&=\left(\overline\ell\pm\Delta\ell\right)^3\\
&=(200mm\pm2mm)^3\\
&\Rightarrow\left\{\begin{array}{l}(2.02\cdot10^{-1})^3m^3\\(1.98\cdot10^{-1})^3m^3\\\end{array}\right.\\
&\Rightarrow\left\{\begin{array}{l}8.24\cdot10^{-3}m^3\\7.76\cdot10^{-3}m^3\\\end{array}\right.\\
\Rightarrow\Delta\ell&=\max(\left|V_1-\overline V\right|,\left|V_2-\overline V\right|)\\
&=0.24\cdot10^{-3}m^3\qed
\end{align}
\aufgabe{5}
\begin{align}
t&=\frac{s}{v}\\
&=\frac{\overline s\pm\Delta s}{c}; c=299792458\frac{m}{s}=2.99792458\cdot10^9\frac{m}{s}=2.99792458\frac{m}{ns}\\
&=\frac{\overline s}{c}\pm\frac{\Delta s}{c}\\
&=\frac{7.312780\cdot10^5nsm}{2.99792458m}\pm\frac{0.2nsm}{2.99792458m}\\
&=\frac{7.312780}{2.99792458}\cdot10^5ns\pm\frac{0.2}{2.99792458}ns\\
&=2.439281\cdot10^5ns\pm6.7\cdot10^{-2}ns\qed
\end{align}
Zu b): Ja, die Abweichung ist signifikant, da sie außerhalb der in der obigen Rechnung bestimmten Fehlergrenzen liegt.
\begin{align}
v&=\frac{s}{t}\\
&=\frac{7.312780\cdot10^5\pm0.2}{243988.8\pm10.2}\cdot\frac{m}{ns}\\
\overline v&=\frac{7.312780\cdot10^5}{2.439888\cdot10^5}\cdot\frac{m}{ns}\\
&=2.9971786 \frac{m}{ns}
\end{align}
\end{document}
