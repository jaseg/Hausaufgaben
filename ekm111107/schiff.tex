\documentclass[12pt,a4paper,notitlepage]{article}
\usepackage[utf8x]{inputenc}
\usepackage[a4paper,textwidth=17cm, top=2cm, bottom=3.5cm]{geometry}
\usepackage{eurosym}
%\usepackage{url}
\usepackage[T1]{fontenc}
\usepackage{ucs}
\usepackage{ngerman} 
\usepackage{setspace}
%\usepackage{fourier}
\usepackage{amssymb,amsmath}
\usepackage{wasysym}
%\usepackage{marvosym}
\usepackage{tabularx}
\usepackage{multicol}
\usepackage{hyperref}
\usepackage[pdftex]{graphicx,color}
\usepackage{todo}
\definecolor{p-green}{rgb}{0.12,0.57,0.11}
\newcommand{\bitem}{\item[--]}
\newcommand{\litem}[2]{\item[#1 --] #2}
\newcommand{\blitem}[3]{\item[#1 --] \texttt{#2} -- #3}
\newcommand{\gfo}{\grqq\ }
\newcommand{\gfu}{\glqq}
\newcommand{\zquote}[2]{\glqq #1\grqq\ (Z.\ #2)}
\newcommand{\pquote}[1]{\glqq #1\grqq}
\newcommand{\nquote}[2]{#1: \glqq #2\grqq}
\newcommand{\nwquote}[3]{#1 -- \emph{#2}: \glqq #3\grqq}
\newcommand{\nwyquote}[4]{#1 -- \emph{#2} (#3): \glqq #4\grqq}
\newcommand{\diff}{\mathrm{d}}
\renewcommand{\abstractname}{}
\definecolor{orange}{rgb}{1,0.6,0}
\definecolor{d-green}{rgb}{0,0.8,0}
\definecolor{pink}{rgb}{1,0,0.6}
\newcommand{\annot}[1]{\textcolor{red}{#1}}
\newcommand{\ecolor}[1]{\textcolor{pink}{#1}}
\newcommand{\re}{\text{Re}}
\newcommand{\im}{\text{Im}}
\onehalfspacing
\setlength{\parskip}{8pt plus4pt minus4pt}
\title{}
\begin{document}
Fortan ist $\ell$ dynamisch: der Anfangswert wird als $s$ bezeichnet. Er errechnet sich wie folgt:
\begin{equation}
s=\sqrt{x_0^2+h^2}
\end{equation}
$v_w$ wird der Übersicht halber mit $v$ abgekürzt.
\begin{align}
x(t)^2+h^2&=l(t)^2\\
l&=s-v\cdot t\\
x&=\sqrt{\left(s-v\cdot t\right)^2-h^2}\\
&=\sqrt{s^2-2svt+t^2v^2-h^2}\\
t&=\frac{\sqrt{x^2+h^2}-x}{v}\label{tofx}
\end{align}
\begin{align}
v_s=\dot x&=\frac{tv^2-sv}{\sqrt{s^2-2svt+t^2v^2-h^2}}\\
&=\frac{tv^2-sv}{x}\\
\text{Nach Einsetzen von \ref{tofx}:}\;&=\frac{v\sqrt{x^2+h^2}-2sv}{x}
\end{align}
\begin{align}
\dot x&=\frac{tv^2-sv}{\sqrt{s^2-2svt+t^2v^2-h^2}}\\
\ddot x&=\frac{\partial\dot x}{\partial t}+\frac{\partial\dot x}{\partial x}\frac{\diff x}{\diff t}\\
&=0+\left(\frac{v\sqrt{x^2+h^2}-2sv}{x}\right)^{(x)}\cdot\dot x\\
&=\dot x\left(\frac{v}{\sqrt{x^2+h^2}}-\frac{v\sqrt{x^2+h^2}-2sv}{x^2}\right)
\end{align}
\end{document}
