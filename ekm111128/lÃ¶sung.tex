\documentclass[12pt,a4paper,notitlepage]{article}
\usepackage[utf8x]{inputenc}
\usepackage[a4paper,textwidth=17cm, top=2cm, bottom=3.5cm]{geometry}
\usepackage{eurosym}
%\usepackage{url}
\usepackage[T1]{fontenc}
\usepackage{ucs}
\usepackage{ngerman} 
\usepackage{setspace}
%\usepackage{fourier}
\usepackage{amssymb,amsmath}
\usepackage{wasysym}
\usepackage{amsthm}
%\usepackage{marvosym}
\usepackage{tabularx}
\usepackage{multicol}
\usepackage{hyperref}
\usepackage{tabularx}
\usepackage[pdftex]{graphicx,color}
\usepackage{todo}
\definecolor{p-green}{rgb}{0.12,0.57,0.11}
\newcommand{\bitem}{\item[--]}
\newcommand{\litem}[2]{\item[#1 --] #2}
\newcommand{\blitem}[3]{\item[#1 --] \texttt{#2} -- #3}
\newcommand{\gfo}{\grqq\ }
\newcommand{\gfu}{\glqq}
\newcommand{\zquote}[2]{\glqq #1\grqq\ (Z.\ #2)}
\newcommand{\pquote}[1]{\glqq #1\grqq}
\newcommand{\nquote}[2]{#1: \glqq #2\grqq}
\newcommand{\nwquote}[3]{#1 -- \emph{#2}: \glqq #3\grqq}
\newcommand{\nwyquote}[4]{#1 -- \emph{#2} (#3): \glqq #4\grqq}
\newcommand{\diff}{\mathrm{d}}
\renewcommand{\abstractname}{}
\definecolor{orange}{rgb}{1,0.6,0}
\definecolor{d-green}{rgb}{0,0.8,0}
\definecolor{pink}{rgb}{1,0,0.6}
\newcommand{\annot}[1]{\textcolor{red}{#1}}
\newcommand{\ecolor}[1]{\textcolor{pink}{#1}}
\newcommand{\aufgabe}[1]{\section*{\setcounter{section}{#1}Aufgabe #1}}
\newcommand{\re}{\text{Re}}
\newcommand{\im}{\text{Im}}
\onehalfspacing
\setlength{\parskip}{8pt plus4pt minus4pt}
\begin{document}
\begin{center}
\Large Hausaufgaben für P1a

\normalsize Online unter \url{http://github.com/jaseg/Hausaufgaben}
\end{center}
\begin{tabularx}{\textwidth}{Xr}
Jan Sebastian Götte (546408), Paul Scheunemann&Abgabe: 111128
\end{tabularx}
\aufgabe{1}
\aufgabe{2}
\subsection*{a)}
\begin{align}
L&=mr_0v_0\\
E&=T+V\\
T&=\frac{1}{2}mv_0^2
\end{align}
Zur Berechnung der potentiellen Energie $V$ muss zunächst die Lage des Equilibriums zwischen Zentrifugalkraft und Schwerkraft bestimmt werden. Doch zunächst ist festzustellen:
\begin{equation}
\dot{\vec L}=\vec 0
\end{equation}
Dies gilt, da die einzige wirkende Kraft radial ist, und dank ihrer Konstanz ein konservatives Kraftfeld bildet. Nun:
\begin{align}
\left|\vec L\right|&=mrv=mr_0v_0\\
\Rightarrow v(r)&=v_0\frac{r_0}{r}\\
F_Z&=-F_R\quad\text{per Voraussetzung}\\
=-m\frac{v(r_t)^2}{r_t}&=-Mg\\
=-m\frac{r_0^2v_0^2}{r_t^3}\\
\Rightarrow r_t&=\sqrt[3]{\frac{mr_0^2v_0^2}{Mg}}\\
\Rightarrow V&=\int_{r_t}^{r_0}F_z\diff r\quad\Big|\dot F_Z=0\\
&=F_Zr\Big|_{r_t}^{r_0}\\
&=Mgr_0-Mg\sqrt[3]{\frac{mr_0^2v_0^2}{Mg}}
\end{align}
Durch Einsetzen erhält man
\begin{equation}
E=\frac{1}{2}mv_0^2+Mgr_0-\sqrt[3]{M^2g^2mr_0^2v_0^2}
\end{equation}
Die Einheitenbetrachtung zeigt keine Fehler auf, Platzhalber sei hier bloß die der Wurzel angeführt:
\begin{equation}
\left[E\right]=J=\left[\sqrt[3]{M^2g^2mr_0^2v_0^2}\right]=\sqrt[3]{kg^2\cdot\frac{m^2}{s^4}\cdot kg\cdot m^2\cdot\frac{m^2}{s^2}}=\frac{kg\cdot m^2}{s^2}
\end{equation}
\subsection*{b)}
\subsection*{c)}
\begin{align}
F_Z&=-F_R=-m\omega^2r\quad\Big|\omega=\frac{v}{r}\\
=-\frac{mv_0^2}{r_0}&=-Mg\\
\Rightarrow\frac{v_0^2}{r_0}&=\frac{M}{m}g
\end{align}
\aufgabe{3}
\begin{align}
\vec F_G&=-\gamma\frac{m_1m_2}{r^2}\frac{\vec r}{r}\\
\Rightarrow F_G&=-\gamma\frac{m_1m_2}{r^2}\\
F_g&=m\cdot g\\
g&=-\gamma\frac{m_E}{r_E^2}\\
\left|1-\gamma\frac{m_E}{r_E^2}\cdot m\cdot\frac{1}{\gamma\frac{m_E}{R^2}\cdot m}\right|&<10^{-3}=0.001\\
\Rightarrow\left|1-\left[\frac{\frac{1}{r_E^2}}{\frac{1}{R^2}}=\frac{R^2}{r_E^2}\right]\right|&<10^{-3}\\
\stackrel{R>r_E}{\Rightarrow}\frac{R^2}{r_E^2}-1&<10^{-3}\\
\Rightarrow R^2&<r_E\cdot\left(1+10^{-3}\right)\\
\Rightarrow R_{\text{max}}&=\sqrt{r_E^2\cdot\left(1+10^{-3}\right)}\\
&=6373km\\
\Rightarrow\Delta R_{\text{max}}&=3km
\end{align}
\aufgabe{5}
\subsection*{a)}
\begin{description}
\item[Energie] bleibt erhalten, da keine Kräfte wirken.
\item[Impuls] bleibt erhalten, da keine Kräfte wirken.
\item[Drehimpuls bezüglich A] bleibt erhalten, da keine Kräfte wirken. $\vec L_A$ ist obendrein $\vec 0$, sofern $A$ auf der gestrichelten Linie liegt.
\item[Drehimpuls bezüglich B] bleibt erhalten, da keine Kräfte wirken.
\end{description}
\subsection*{b)}
\begin{description}
\item[Energie] bleibt erhalten, da $\vec F$ konservativ ist:
\begin{align}
\vec F(\vec r)&=f\left(\left|\vec r\right|\right)\cdot\vec e_r\\
\vec\nabla\times\vec F\left(\vec r\right)&=\vec 0
\end{align}
Letzteres folgt daraus, dass $f$ bloß von $\left|\vec r\right|$ abhängt.
\item[Impuls] bleibt nicht erhalten, da Kräfte wirken ($\vec F=\dot\vec p\neq\vec 0$).
\item[Drehimpuls bezüglich A] bleibt nicht erhalten, da $\frac{\diff\vec L}{\diff t}=\vec M=\vec x\times\vec F$ nicht verschwindet ($\vec F$ ist unabhängig von $\vec x$).
\item[Drehimpuls bezüglich B] bleibt erhalten, da $\frac{\diff\vec L}{\diff t}=\vec M=\vec x\times\vec F=\vec 0$, da $\vec x$ und $\vec F$ per Definition kollinear sind.
\end{description}
\end{document}
